%%%%%%%%%%%%%%%%%%%%%%%%%%%%%%%%%%%%%%%%%
% The Legrand Orange Book
% LaTeX Template
% Version 2.3 (8/8/17)
%
% This template has been downloaded from:
% http://www.LaTeXTemplates.com
%
% Original author:
% Mathias Legrand (legrand.mathias@gmail.com) with modifications by:
% Vel (vel@latextemplates.com)
%
% License:
% CC BY-NC-SA 3.0 (http://creativecommons.org/licenses/by-nc-sa/3.0/)
%
% Compiling this template:
% This template uses biber for its bibliography and makeindex for its index.
% When you first open the template, compile it from the command line with the 
% commands below to make sure your LaTeX distribution is configured correctly:
%
% 1) pdflatex main
% 2) makeindex main.idx -s StyleInd.ist
% 3) biber main
% 4) pdflatex main x 2
%
% After this, when you wish to update the bibliography/index use the appropriate
% command above and make sure to compile with pdflatex several times 
% afterwards to propagate your changes to the document.
%
% This template also uses a number of packages which may need to be
% updated to the newest versions for the template to compile. It is strongly
% recommended you update your LaTeX distribution if you have any
% compilation errors.
%
% Important note:
% Chapter heading images should have a 2:1 width:height ratio,
% e.g. 920px width and 460px height.
%
%%%%%%%%%%%%%%%%%%%%%%%%%%%%%%%%%%%%%%%%%

%----------------------------------------------------------------------------------------
%	PACKAGES AND OTHER DOCUMENT CONFIGURATIONS
%----------------------------------------------------------------------------------------

\documentclass[11pt,fleqn]{book} % Default font size and left-justified equations
%----------------------------------------------------------------------------------------

\input{../structure} % Insert the commands.tex file which contains the majority of the structure behind the template
\usepackage{graphicx}
\usepackage{lscape}
\usepackage{pgfgantt}
\usepackage{multicol}
\usepackage{caption} 
\usepackage{float}
\usepackage{pdfpages}
\usepackage{wrapfig}
\usepackage{enumitem}
\usepackage[export]{adjustbox}
\captionsetup[table]{skip=5pt}
\usepackage{parskip}  %Prevents having to use // after each paragraph
\usepackage{todonotes} %For easy Todo managment

\pdfinclusioncopyfonts=1 % This fixes missing characher form figures
\pdfsuppresswarningpagegroup=1 % ignore page group warning

\begin{document}	
	\setlength{\marginparwidth}{2cm}
	\thispagestyle{empty}
	\makeatletter\@openrightfalse
	\makeatother

	% Overwrite the page-number to disable it
	\fancyfoot[LE,RO]{}
	% Disable the use of a chapter image
	\usechapterimagefalse
	
	\pagestyle{empty}
	
	% Make a new chapter heading for the cards
	\renewcommand{\chapter}[1]{
	\newpage
	\begin{tikzpicture}[remember picture,overlay]
		\node at (current page.north west){
			\begin{tikzpicture}[remember picture,overlay]
				\textbf{\draw[anchor=west] (0cm,-1cm) node [fill=black,fill opacity=0.4,inner sep=12pt]{\strut\makebox[22cm]{}};}
				\draw[anchor=west] (1.5cm,-1.1cm) node {\Huge\sffamily\bfseries\color{white}#1};
				\draw[anchor=east] (19.5cm,-1cm) node {\Huge\sffamily\bfseries\color{white}KB II};
		\end{tikzpicture}};
	\end{tikzpicture}
	}
	
%	\fancyfoot[C]{Concept versie}
	
	\chapter{Oploeven}
	\begin{tikzpicture}[remember picture,overlay]
		\node[inner sep=0pt] (background) at (current page.center) {\includegraphics[width=\paperwidth, page = 1]{../Hoofdstukken/Manoeuvres/manoeuvres.pdf}};
	\end{tikzpicture}


	\chapter{Afvallen}
	\begin{tikzpicture}[remember picture,overlay]
		\node[inner sep=0pt] (background) at (current page.center) {\includegraphics[width=\paperwidth, page = 2]{../Hoofdstukken/Manoeuvres/manoeuvres.pdf}};
	\end{tikzpicture}


	\chapter{Overstag}
	\begin{tikzpicture}[remember picture,overlay]
		\node[inner sep=0pt] (background) at (current page.center) {\includegraphics[width=0.95\paperwidth, page = 3]{../Hoofdstukken/Manoeuvres/manoeuvres.pdf}};
	\end{tikzpicture}
	
	\chapter{Gijp}
	\begin{tikzpicture}[remember picture,overlay]
		\node[inner sep=0pt] (background) at (current page.center) {\includegraphics[width=0.95\paperwidth, page = 4]{../Hoofdstukken/Manoeuvres/manoeuvres.pdf}};
	\end{tikzpicture}

	\chapter{Hogerwal}

	\begin{tikzpicture}[remember picture,overlay]
		\node[inner sep=0pt] (background) at (current page.center) {\includegraphics[width=\paperwidth, page = 5]{../Hoofdstukken/Manoeuvres/manoeuvres.pdf}};
	\end{tikzpicture}
	\newpage
	\begin{tikzpicture}[remember picture,overlay]
		\node[inner sep=0pt] (background) at (current page.center) {\includegraphics[width=\paperwidth, page = 6]{../Hoofdstukken/Manoeuvres/manoeuvres.pdf}};
	\end{tikzpicture}


	\chapter{Man over boord}

	\begin{tikzpicture}[remember picture,overlay]
		\node[inner sep=0pt] (background) at (current page.center) {\includegraphics[width=0.95\paperwidth, page = 7]{../Hoofdstukken/Manoeuvres/manoeuvres.pdf}};
	\end{tikzpicture}
	

	\chapter{Stormrondje}
	\begin{tikzpicture}[remember picture,overlay]
		\node[inner sep=0pt] (background) at (current page.center) {\includegraphics[width=0.95\paperwidth, page = 8]{../Hoofdstukken/Manoeuvres/manoeuvres.pdf}};
	\end{tikzpicture}

\end{document}
