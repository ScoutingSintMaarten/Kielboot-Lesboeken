\header{0}
\chapter*{Voorwoord}

\section{Voorwoord}
Om ons huidige, inmiddels flink verouderde, kielboot theorieboek te vervangen ben ik begin 2018 begonnen met het schrijven van deze vernieuwde versie. Hoewel het oude boek zeker niet slecht was, was het duidelijk tijd voor wat vernieuwing. Qua inhoud is dit boek erg geïnspireerd op zijn voorloper, met de onderscheidende factor van duidelijkere afbeeldingen en illustraties, verbeterde teksten en een modern thema.

- Christian Peppelman
\section{Dankwoord}
Tijdens het opstellen van dit boek heb ik heel erg veel waardevolle feedback mogen ontvangen van medeleden van onze scoutingvereniging. Ik wil hen daar hartelijk voor bedanken! In het bijzonder Yara, Robert en Wouter voor de kritische, maar opbouwende feedback die dit boek gemaakt heeft tot wat het nu is.\\
Daarnaast wil ik ook graag de Katwijkse Zeeverkenners bedanken voor het online beschikbaar stellen van hun uitstekende lesboeken (\url{https://www.katwijksezeeverkenners.nl/cwo/instructieboeken/}). Het lesboek van de Katwijkse Zeeverkenners is een grote inspiratiebron geweest voor de figuren in dit lesboek.

\newpage
\section{Lesstof verantwoording}
De lesstof die in dit boek aan bod komt, is gemaakt om aan de eisen van de stichting Commissie Watersport Opleidingen (CWO) te voldoen voor de discipline kielboot II. Deze eisen zijn te vinden op \url{https://cwo.nl/leren-varen/kielboot}. Op sommige vlakken gaat dit boek uitgebreider in op de stof dan vanuit het CWO strikt noodzakelijk is. Hiervoor is gekozen omdat deze kennis een toegevoegde waarde kan bieden tijdens het zeilen op scouting.


\section{Document Informatie}
\subsection*{Licentie}
\begin{figure}[H]
	\centering
	\begin{minipage}[t]{0.60\textwidth}
		\vspace{-1.80cm}
		Dit boek is uitgebracht onder een Creative Commons
		'Naamsvermelding-NietCommercieel-GelijkDelen 4.0 Internationaal' (CC BY-NC-SA 4.0) licentie. Voor meer informatie: \url{https://creativecommons.org/licenses/by-nc-sa/4.0/}
	\end{minipage}
	\hfill
	\begin{minipage}[b]{0.35\textwidth}
	\includegraphics[width=\textwidth]{../Hoofdstukken/Informatie/CC-BY-NC-SA.png}
\end{minipage}
\end{figure}
\subsection*{Auteur informatie}
Dit boek is geschreven door Christian Peppelman.\\ 
Voor contact, vragen of verbeteringen kun je mailen naar: \href{mailto:cwo@sintmaartengroep.nl}{CWO@sintmaartengroep.nl} 
\subsection*{Gebruik}
Om optimaal gebruik te kunnen maken van dit lesboek, deze graag laten drukken in een geniete brochure in kleur. Gelieve het boek niet thuis te printen, inscannen of vermenigvuldigen in een manier die negatieve invloed op de kwaliteit heeft. Voor de originele bestanden of gedrukte varianten kun je contact opnemen of kijken op \url{https://sintmaartengroep.nl/}
\subsection*{Thema}
Het thema waar dit boek op gebaseerd is heet `The Legrand Orange Book' en is ontworpen door Mathias Legrand. Het thema is gedownload op \url{https://nl.overleaf.com/latex/templates/} en valt onder een Creative Commons BY-NC-SA 3.0 licentie.
\subsection*{Versiebeheer}
\begin{table}[H]
	\centering
	\begin{tabular}{c|l|p{8cm}}
		\textbf{Versie} & \textbf{Datum} & \textbf{Omschrijving} \\ \hline
		1.5 & 9 januari 2019 & Eerste druk  \\ \hline
	    1.6 & 23 augustus 2019 & Toevoeging Deel III: Zeilmanoeuvres  \\ \hline
		1.7 & 24 september 2019  & Spelling verbeteringen \\ \hline
		2.0 & 5 januari 2020  & Afronding versie 2 \\ \hline
		2.1 & 21 maart 2020  & Kleine verbeteringen \\ \hline
		2.2 & 23 maart 2021  & Figuur 2.5 vernieuwd en toevoeging antwoordenblad \\ \hline	
		2.3 & 6 november 2023  & Update hoofdstuk reglementen
	\end{tabular}
\end{table}


\textit{Versie 2.3 \hspace{1 cm} 6 november 2023}
%Druk verhoogt alleen met 0.x versie verhogingen of hoger