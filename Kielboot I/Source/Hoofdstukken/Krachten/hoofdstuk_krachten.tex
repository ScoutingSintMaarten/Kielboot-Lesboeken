\header{7}
\chapter{Krachten op het schip}
\section{Inleiding}
De wind is de kracht die een zeilboot vooruit laat gaan. In dit hoofdstuk wordt gekeken naar het effect van deze kracht op de fok en het grootzeil. Daarnaast wordt besproken hoe je je zeil correct kunt zetten aan de hand van de wind.


\section{Effecten van de fok en het grootzeil}
Het grootzeil en de fok hebben allebei een ander effect op de boot. Als je alleen de fok aantrekt gaat je boot afvallen zoals in figuur \ref{pic:effect_fok}. Het tegenovergestelde gebeurt als je alleen het grootzeil aantrekt. In dit geval loef je op als in figuur \ref{pic:effect_grootzeil}.
\begin{figure}[ht]
  \centering
  \begin{minipage}[b]{0.49\textwidth}
  \centering
    \includegraphics[width=0.6\textwidth]{Hoofdstukken/Krachten/pdf/effect_fok.pdf}
    \caption{Effect van de fok}
    \label{pic:effect_fok}
  \end{minipage}
  \hfill
  \begin{minipage}[b]{0.49\textwidth}
    \centering
    \includegraphics[width=0.6\textwidth]{Hoofdstukken/Krachten/pdf/effect_grootzeil.pdf}
    \caption{Effect van het grootzeil}
    \label{pic:effect_grootzeil}
    \end{minipage}
\end{figure}
\section{Correcte zeilstand}
\begin{figure}[H]
	\centering
	\begin{minipage}[t]{0.67\textwidth}
		Het is belangrijk om de stand van je zeil aan te passen op de koers die je vaart. Door je zeil op de juiste stand te zetten kan je zo snel mogelijk varen. In figuur  \ref{pic:zeil_goed} is een grootzeil te zien wat goed is aangetrokken.\\ 
		
		Als je het grootzeil te strak aantrek, kan deze bij het achterlijk (de achterkant van het zeil) gaan klapperen.
		
	\end{minipage}
	\hfill
	\begin{minipage}[t]{0.15\textwidth}
		\raisebox{-0.9\height}{\includegraphics[width=\textwidth]{Hoofdstukken/Krachten/pdf/zeil_goed.pdf}}
		\caption{}
		\label{pic:zeil_goed}
	\end{minipage}
	\hfill
	\begin{minipage}[t]{0.15\textwidth}
		\raisebox{-0.9\height}{\includegraphics[width=\textwidth]{Hoofdstukken/Krachten/pdf/zeil_los.pdf}}
		\caption{}
		\label{pic:zeil_los}
	\end{minipage}
\end{figure}


 Wanneer je het grootzeil te los zet, zal deze gaan tegenbollen bij het voorlijk (de voorkant van het zeil). In figuur \ref{pic:zeil_los} zie je een zeil dat tegenbolt. Dit remt je heel erg af.  


Een goede manier om je zeil te stellen, is om hem net zo lang te laten vieren, totdat je een kleine tegenbolling ziet in het voorlijk. Dan trek je het zeil weer een klein beetje aan. Op deze manier weet je zeker dat je zeil goed staat.

\section{Conclusie}
In dit hoofdstuk zijn een aantal krachten op het schip behandeld. Je hebt geleerd dat je met de fok afvalt en met het grootzeil oploeft. Je kunt ook herkennen wanneer het grootzeil goed of fout staat.