\header{5}
\chapter{Reglementen \& Voorrangsregels}
\section{Inleiding}
In dit hoofdstuk worden de regels en wetten op het water uitgelegd. Op de meeste wateren in Nederland wordt gebruik gemaakt van het Binnenvaartpolitiereglement, het BPR. Het BPR bevat alle regels over hoe je met elkaar om moet gaan op het water.

\section{Algemene reglementen}
Om het BPR goed te kunnen begrijpen, zullen er eerst een aantal algemene zaken besproken worden. Als eerste vijf definities van boten: motorschip, zeilschip, roeiboot, klein schip en groot schip. 

\begin{itemize}
    \item \textbf{Motorschip:} Een schip dat zich met een motor voortbeweegt.\hfill \raisebox{-0.4\height}{\includegraphics[height=1cm]{Hoofdstukken/Reglementen/pdf/motorboot.pdf}}
    \item \textbf{Zeilschip:} Een schip dat \textbf{alleen} zijn zeilen gebruikt om voort te bewegen.\hfill \raisebox{-0.4\height}{\includegraphics[height=1cm]{Hoofdstukken/Reglementen/pdf/zeilboot.pdf}}
    \item \textbf{Roeiboot:} Een schip dat \textbf{alleen} spierkracht gebruikt om zich voort te bewegen.\hfill \raisebox{-0.4\height}{\includegraphics[height=1cm]{Hoofdstukken/Reglementen/pdf/roeiboot.pdf}}
    \item \textbf{Klein schip:} Alle schepen onder de 20 meter.
    \item \textbf{Groot schip:} Schepen groter dan 20 meter.\hfill \raisebox{-0.4\height}{\includegraphics[height=1cm]{Hoofdstukken/Reglementen/pdf/groot_schip.pdf}}
\end{itemize}

\subsection{Goed zeemanschap}
Het goed zeemanschap is een hele belangrijke regel op het water. Deze regel houdt in dat de schipper er \textbf{alles} aan moet doen om een aanvaring te voorkomen. Dit betekent ook dat een schipper voor eigen veiligheid of die van anderen mag afwijken van de regels op het water.


\section{Voorrangsregels}
Voorrangssituaties zijn te verdelen in drie types: kruisende koersen, tegengestelde koeren en oplopende koersen. Deze drie koersen zijn te zien in figuur \ref{pic:voorrangkoers} en worden ook wel de voorrangskoersen genoemd. De koers die je vaart bepaalt met welke voorrangsregels je te maken hebt. 
\begin{figure}[H]
    \centering
    \includegraphics[width=1\textwidth]{Hoofdstukken/Reglementen/pdf/voorrangskoersen.pdf}
    \caption{Voorrangskoersen}
    \centering
    \label{pic:voorrangkoers}
\end{figure}
\vfil\newpage
De voorrangsregels hebben een volgorde. Je kijkt altijd eerst naar de bovenste regel. Als deze regel niet van toepassing is, ga je pas door naar de volgende. Dit doe je net zo lang totdat er een regel is die toe te passen is op jouw situatie.


\subsection{Kruisende Koersen}
Bij kruisende koersen kijk je naar de volgende voorrangsregels:
\begin{figure}[H]
	\centering
	\begin{minipage}[t]{0.70\textwidth}
	\textbf{1.} Het schip dat aan de stuurboordswal vaart heeft voorrang.\\
	\textit{Zeilschip A vaart aan de stuurboordswal en heeft voorrang op B}
	\end{minipage}
	\hfill
	\begin{minipage}[t]{0.20\textwidth}
		\raisebox{-0.5\height}{\includegraphics[width=\textwidth]{Hoofdstukken/Reglementen/pdf/kruis_stuurboordswal.pdf}}
		\label{pic:kr1}
	\end{minipage}
	\hfill
\end{figure}

\vspace{-0.7cm}

\begin{figure}[H]
	\centering
	\begin{minipage}[t]{0.70\textwidth}
	\textbf{2.} Grote schepen hebben voorrang op kleine schepen.\\
		\textit{Het grote motorschip A heeft voorrang op het kleine motorschip B}
	\end{minipage}
	\hfill
	\begin{minipage}[t]{0.20\textwidth}
	\raisebox{-0.5\height}{\includegraphics[width=\textwidth]{Hoofdstukken/Reglementen/pdf/kruis_groot_klein.pdf}}	
	\label{pic:kr2}
	\end{minipage}
	\hfill
\end{figure}

\vspace{-0.7cm}
\begin{figure}[H]
	\centering
	\begin{minipage}[t]{0.70\textwidth}
		\textbf{3.} Een zeilschip gaat voor een roeiboot gaat voor een motorschip.\\
		\textit{Zeilschip A heeft voorrang op roeiboot B en motorschip C\\
			 Roeiboot B heeft voorrang op motorschip C}
	\end{minipage}
	\hfill
	\begin{minipage}[t]{0.20\textwidth}
		\raisebox{-0.65\height}{\includegraphics[width=\textwidth]{Hoofdstukken/Reglementen/pdf/kruis_zsm.pdf}}	
		\label{pic:kr3}
	\end{minipage}
	\hfill
\end{figure}

\textbf{4.} Bij zeilschepen onderling zijn de volgende twee regels van belang:
\begin{figure}[H]
	\centering
	\hspace{0.02\textwidth}
	\begin{minipage}[t]{0.70\textwidth}
		\textbf{4.1}. Een zeilschip met zeilen over bakboord heeft voorrang.\\
		\textit{Zeilschip B (met zijn zeilen over bakboord) heeft voorrang op \\zeilschip A (met zijn zeilen over stuurboord)}
	\end{minipage}
	\hfill
	\begin{minipage}[t]{0.20\textwidth}
		\raisebox{-0.6\height}{\includegraphics[width=\textwidth]{Hoofdstukken/Reglementen/pdf/kruis_zeilboot_onderling_bakboord.pdf}}	
		\label{pic:kr41}
	\end{minipage}
	\hfill
\end{figure}

\vspace{-0.7cm}

\begin{figure}[H]
	\centering
	\hspace{0.02\textwidth}
	\begin{minipage}[t]{0.70\textwidth}
		\textbf{4.2.} Een zeilschip aan lij heeft voorrang op een zeilschip aan loef.\\
		\textit{Zeilschip A ligt aan loef van zeilschip B en verleent dus voorrang}
	\end{minipage}
	\hfill
	\begin{minipage}[t]{0.20\textwidth}
		\raisebox{-0.55\height}{\includegraphics[width=\textwidth]{Hoofdstukken/Reglementen/pdf/kruis_zeilboot_onderling_loef_lij.pdf}}	
		\label{pic:kr42}
	\end{minipage}
	\hfill
\end{figure}


\subsection{Tegengestelde Koersen}
Bij tegengestelde koersen kijk je naar de volgende voorrangsregels:
\begin{figure}[H]
	\centering
	\begin{minipage}[t]{0.70\textwidth}
		\textbf{1.} Het schip dat aan de stuurboordswal vaart heeft voorrang\\
		\textit{Zeilschip B vaart aan de stuurboordswal en heeft voorrang op A}
	\end{minipage}
	\hfill
	\begin{minipage}[t]{0.25\textwidth}
		\raisebox{-0.55\height}{\includegraphics[width=\textwidth]{Hoofdstukken/Reglementen/pdf/tegen_stuurboord.pdf}}
		\label{pic:tg1}
	\end{minipage}
	\hfill
\end{figure}

\begin{figure}[H]
	\centering
	\begin{minipage}[t]{0.70\textwidth}
		\textbf{2.} Grote schepen hebben voorrang op kleine schepen.\\
		\textit{Het grote motorschip B heeft voorrang op het kleine motorschip A}
	\end{minipage}
	\hfill
	\begin{minipage}[t]{0.25\textwidth}
		\raisebox{-0.55\height}{\includegraphics[width=\textwidth]{Hoofdstukken/Reglementen/pdf/tegen_groot_klein.pdf}}
		\label{pic:tg2}
	\end{minipage}
	\hfill
\end{figure}


\subsection{Oplopende koersen}
\begin{figure}[H]
	\centering
	\begin{minipage}[t]{0.70\textwidth}
		Bij het oplopen of inhalen, wijkt het oplopende schip. Het schip dat opgelopen wordt kan uitwijken als het nodig is.\\
		\textit{Zeilschip A (oploper) wijkt om op te lopen. Zeilschip B wijkt om wat extra ruimte te maken.}
	\end{minipage}
	\hfill
	\begin{minipage}[t]{0.25\textwidth}
		\raisebox{-0.8\height}{\includegraphics[width=\textwidth]{Hoofdstukken/Reglementen/pdf/oplopen.pdf}}
		\label{pic:op}
	\end{minipage}
	\hfill
\end{figure}

\subsection{Toevoegingen}
Bij zeilboten onderling is het belangrijk dat je, als het kan, altijd aan de loefzijde inhaalt. In figuur \ref{pic:oplopen_lij} zie je dat zeilschip B ingehaald wordt door zeilschip A aan de loefzijde. Hiermee neemt zeilschip A de wind uit de zeilen van zeilschip B. Het oplopen gaat hierdoor sneller.

Bij het oplopen aan de lijzijde vangt de oploper juist minder wind. Dit is te zien bij zeilschip B in figuur \ref{pic:oplopen_lij}. Hierdoor gaat het oplopen minder snel. Soms lukt het zelfs helemaal niet om op te lopen aan de lijzijde.

\begin{figure}[h]
	\centering
	\begin{minipage}[b]{0.49\textwidth}
		\centering
		\includegraphics[width=0.65\textwidth]{Hoofdstukken/Reglementen/pdf/oplopen_loef.pdf}
		\caption{Oplopen aan loef}
		\centering
		\label{pic:oplopen_loef}
	\end{minipage}
	\hfill
	\begin{minipage}[b]{0.49\textwidth}
		\centering
		\includegraphics[width=0.65\textwidth]{Hoofdstukken/Reglementen/pdf/oplopen_lij.pdf}
		\caption{Oplopen aan lij}
		\label{pic:oplopen_lij}
	\end{minipage}
\end{figure}
	
		
Daarnaast is het belangrijk te onthouden dat wanneer je vaarwater oversteekt, je geen voorrang hebt. Andere schepen moeten hun koers en snelheid niet of nauwelijks hoeven aan te passen voor jouw manoeuvre. 

\subsection{Voorrangsregels op een rij}
Om de voorrangsregels makkelijk te kunnen onthouden staan ze hieronder samengevat:\\\\
Bij \textbf{kruisende koersen} kijk je naar de volgende regels:
\vspace*{-0.15cm}
\begin{enumerate}
	\item Het stuurboordswal varende schip gaat voor
	\item Grote schepen gaan voor op kleine schepen
	\item Zeilboot gaat voor roeiboot gaat voor motorboot
	\item Zeilboten onderling: 
	
	\begin{enumerate}
		\item [4.1.]Zeilen over bakboord gaat voor
		\item [4.2.]Loef wijkt voor lij
	\end{enumerate}
\end{enumerate}

Bij \textbf{tegengestelde koersen} kijk je naar de volgende regels:
\vspace*{-0.15cm}
\begin{enumerate}
	\item Het stuurboordswal varende schip gaat voor
	\item Grote schepen gaan voor op kleine schepen
\end{enumerate}

Bij \textbf{oplopende koersen} wijkt de oploper. Het opgelopen schip kan indien nodig uitwijken.


\section{Conclusie}
Na het lezen van dit hoofdstuk heb je kennis van de voorrangsregels op het water. Een van de belangrijkste is het goed zeemanschap, wat inhoudt dat je alles doet om een gevaarlijke situatie of aanvaring te voorkomen. Daarnaast ken je de verschillende voorrangssituaties en volgorde van de voorrangsregels en weet je hoe je deze moet toepassen. 