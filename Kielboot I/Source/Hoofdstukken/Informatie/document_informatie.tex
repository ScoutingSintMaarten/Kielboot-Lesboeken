\header{0}
\chapter*{Voorwoord}

\section{Voorwoord}
Nadat het eerder gemaakte kielboot II lesboek twee jaar succesvol binnen mijn scoutinggroep werd gebruikt, was het maken van een kielboot I versie voor de hand liggend. Het lesboek lijkt erg op de kielboot II variant, maar bevat enkel de voor kielboot I relevante kennis. Het taalgebruik is ook versimpeld zodat het boek toegankelijker is voor jongere kielboot I cursisten.

- Christian Peppelman
\section{Dankwoord}
Het schrijven van dit boek zou niet mogelijk geweest zijn zonder de hulp en feedback van de mensen om mij heen. Ik wil in het bijzonder Jasmijn bedanken voor haar hulp en motivatie tijdens het maken van dit boek.

Daarnaast wil ik ook graag de Katwijkse Zeeverkenners bedanken voor het online beschikbaar stellen van hun uitstekende lesboeken (\url{https://www.katwijksezeeverkenners.nl/cwo/instructieboeken/}). Het lesboek van de Katwijkse Zeeverkenners is een grote inspiratiebron geweest voor de figuren in dit lesboek.


\section{Lesstof verantwoording}
De lesstof die in dit boek aan bod komt, is gemaakt om aan de eisen van de stichting Commissie Watersport Opleidingen (CWO) te voldoen voor de discipline kielboot I. Deze eisen zijn te vinden op \url{https://cwo.nl/leren-varen/kielboot}. Op sommige vlakken gaat dit boek uitgebreider in op de stof dan vanuit het CWO strikt noodzakelijk is. Hiervoor is gekozen omdat deze kennis een toegevoegde waarde kan bieden tijdens het zeilen op scouting.

\vfil\newpage

\section{Document Informatie}
\subsection*{Licentie}
\begin{figure}[H]
	\centering
	\begin{minipage}[t]{0.60\textwidth}
		\vspace{-1.80cm}
		Dit boek is uitgebracht onder een Creative Commons
		'Naamsvermelding-NietCommercieel-GelijkDelen 4.0 Internationaal' (CC BY-NC-SA 4.0) licentie. Voor meer informatie: \url{https://creativecommons.org/licenses/by-nc-sa/4.0/}
	\end{minipage}
	\hfill
	\begin{minipage}[b]{0.35\textwidth}
	\includegraphics[width=\textwidth]{../Hoofdstukken/Informatie/CC-BY-NC-SA.png}
\end{minipage}
\end{figure}
\subsection*{Auteur informatie}
Dit boek is geschreven door Christian Peppelman.\\ 
Voor contact, vragen of verbeteringen kun je mailen naar: \href{mailto:cwo@sintmaartengroep.nl}{CWO@sintmaartengroep.nl} 
\subsection*{Gebruik}
Om optimaal gebruik te kunnen maken van dit lesboek, deze graag laten drukken in een geniete brochure in kleur. Gelieve het boek niet thuis te printen, inscannen of vermenigvuldigen in een manier die negatieve invloed op de kwaliteit heeft. Voor de originele bestanden of gedrukte varianten kun je contact opnemen of kijken op \url{https://sintmaartengroep.nl/}
\subsection*{Thema}
Het thema waar dit boek op gebaseerd is heet `The Legrand Orange Book' en is ontworpen door Mathias Legrand. Het thema is gedownload op \url{https://nl.overleaf.com/latex/templates/} en valt onder een Creative Commons BY-NC-SA 3.0 licentie.
\subsection*{Versiebeheer}
\begin{table}[H]
	\centering
	\begin{tabular}{c|l|p{8cm}}
		\textbf{Versie} & \textbf{Datum} & \textbf{Omschrijving} \\ \hline
		0.1 & 26 februari 2021 & Eerste concept  \\ \hline
		0.2 & 26 maart 2021 & H1 en H3 verbeterd \\ \hline
		0.3 & 21 mei 2021 & Spelling verbeteringen \\ \hline
		1.0 & 27 mei 2021 & Eerste versie \\ \hline
		1.1 & 30 december 2025 & Spellingverbeteringen en correct loef en lij voorbeeld
	\end{tabular}
\end{table}



\textit{Versie 1.1 \hspace{1 cm} 30 december 2025 \hspace{1cm} Druk 1}
%Druk verhoogt alleen met 0.x versie verhogingen of hoger