\header{2}
\chapter{Bootonderdelen \& Zeiltermen}
\section{Inleiding}
In dit hoofdstuk komen de verschillende onderdelen van de boot en een aantal zeiltermen aan bod. Deze termen en onderdelen zijn belangrijk om de volgende hoofdstukken in dit boek goed te begrijpen.

\section{Zeiltermen}
Voor duidelijke communicatie tijdens de les en in de boot is het van belang dat je een aantal zeiltermen kent. De belangrijkste termen worden hieronder besproken.

\subsection{Bakboord, Stuurboord, Loef en Lij}
Op het water wordt geen gebruik gemaakt van links en rechts, maar bakboord en stuurboord. Bakboord is links en stuurboord is rechts. Om het bakboord en stuurboord van de boot te bepalen kijk je altijd met de vaarrichting mee. Dus van het achterdek naar het voordek. 

Loef en lij zeggen iets over de wind ten opzichte van je boot. De kant waar de wind de boot in komt, is de loefzijde, ook wel de hoge kant genoemd. De kant waar de wind de boot verlaat heet de lijzijde of lage kant. 
\begin{figure}[ht]
	\centering
	\includegraphics[width=0.9\textwidth]{Hoofdstukken/Onderdelen/pdf/wallen.pdf}
	\caption{Hoger- en Lagerwal}
	\centering
	\label{pic:hoog_laag}
\end{figure}

De hogerwal is de wal waar de wind vandaan komt. De lagerwal is de wal waar de wind naar toe waait. Al deze termen zijn te zien in figuur \ref{pic:hoog_laag}

\vfil\newpage

\subsection{Koersen}
Een koers vertelt iets over hoe je boot ligt ten opzichte van de wind. Alle koersen kun je zowel over bakboord, als stuurboord varen, behalve in de wind. Een overzicht van de koersen is te zien in figuur \ref{pic:koersen}. Voor het uitvoeren van veel zeilmanoeuvres is het nodig om de juiste koers te varen. Het is daarom belangrijk om deze koersen goed te kennen en te kunnen varen. 

\begin{figure}[h]
	\centering
	\includegraphics[width=0.9\textwidth]{Hoofdstukken/Onderdelen/pdf/koersen.pdf}
	\caption{Windkoersen}
	\label{pic:koersen}
\end{figure}
 Wanneer je van koers verandert kun je twee kanten op draaien met de voorkant van de boot: naar de wind toe of van de wind af. Het naar de wind toe draaien wordt ook wel oploeven genoemd en is te zien in figuur \ref{pic:oploeven}. Het wegdraaien van de wind wordt afvallen genoemd en is te zien in figuur \ref{pic:afvallen}.
 
 \begin{figure}[H]
	\centering
	\begin{minipage}[b]{0.40\textwidth}
	\includegraphics[width=\textwidth]{Hoofdstukken/Onderdelen/pdf/oploeven.pdf}
	
	\caption{Oploeven}
	\label{pic:oploeven}
	\end{minipage}
	\hspace{2cm}
	\begin{minipage}[b]{0.40\textwidth}
		\includegraphics[width=\textwidth]{Hoofdstukken/Onderdelen/pdf/afvallen.pdf}
		
		\caption{Afvallen}
		\label{pic:afvallen}
	\end{minipage}
	\hfill
\end{figure}

\newpage


\subsection{Overstag}
Wanneer je overstag gaat, ga je van aan de wind over de ene boeg (kant van de boot) naar aan de wind over de andere boeg. Een voorbeeld hiervan is te zien in figuur \ref{pic:overstag}: De boot gaat hier van aan de wind over stuurboord naar aan de wind over bakboord. Een overstag wordt ook wel een `wending' genoemd.


\subsection{Opkruisen}
Opkruisen gebruik je wanneer je naar een punt wil zeilen wat in de wind ligt. Je doet dit door telkens aan de wind te gaan varen en overstag te gaan. Een voorbeeld hiervan is te zien in figuur \ref{pic:opkruisen}. 
\begin{figure}[h]
  \centering
    \begin{minipage}[b]{0.29\textwidth}
  	\centering
  	\includegraphics[width=0.9\textwidth]{Hoofdstukken/Onderdelen/pdf/overstag.pdf}
  	\caption{Overstag}
  	\label{pic:overstag}
  \end{minipage}
  \hfill
  \begin{minipage}[b]{0.29\textwidth}
	\centering
	\includegraphics[width=0.9\textwidth]{Hoofdstukken/Onderdelen/pdf/opkruisen.pdf}
	\caption{Opkruisen}
	\label{pic:opkruisen}
  \end{minipage}
  \hfill
  \begin{minipage}[b]{0.29\textwidth}
    \centering
    \includegraphics[width=0.9\textwidth]{Hoofdstukken/Onderdelen/pdf/gijpen.pdf}
    \caption{Gijpen}
    \label{pic:gijp}
  \end{minipage}
\end{figure}

\subsection{Gijpen}
Bij een gijp ga je van voor de wind over de ene boeg naar voor de wind over de andere boeg. In figuur \ref{pic:gijp} is een boot te zien die gijpt van voor de wind over bakboord naar voor de wind over stuurboord.

\begin{center}\textit{De overstag en gijp worden in deel \ref{part:manoeuvre} `Zeilmanoeuvres' stap voor stap uitgelegd.}
\end{center}


\newpage

\section{Bootonderdelen}
In figuur \ref{pic:vlet_nummers} is een tekening van een lelievlet te zien met maar liefst 88 genummerde onderdelen. De namen van de onderdelen staan in tabel \ref{table:vletwel}. Alle onderdelen met een blauw nummer moet je kennen.

\begin{figure}[h!]
	\centering
	\makebox[\textwidth][c]{\includegraphics[width=1.2\textwidth]{Hoofdstukken/Onderdelen/png/lelievlet_onderdelen.png}}
	\caption{Tekening lelievlet met nummers \protect\footnotemark}
	\centering
	\label{pic:vlet_nummers}
\end{figure}

\footnotetext{\textit{Lelievlet\_onderdelen.png}, https://www.willibrordusgroep.nl/Images/upload/cwo/lelievlet\_onderdelen.png, Feb 2021.
}


\begin{table}[h!]
	\centering
	\caption{Vletonderdelen}
	
	\setlength\extrarowheight{5pt} %Add height to center text vertically
	\renewcommand{\arraystretch}{0.75} %Shrink total heigt to keep row same heigt
	\newcommand{\tabhead}[1]{\cellcolor{ocre}{\color[HTML]{FFFFFF}\sffamily \textbf{#1}}}
	\newcommand{\NIL}[1]{\cellcolor{not}{#1}}
	\newcommand{\WIL}[1]{\cellcolor{wel}{#1}}
	\label{table:vletwel}
	
	\begin{tabular}{|ll|ll|ll|ll|}
	\multicolumn{2}{|l|}{\tabhead{Fok}}       & \multicolumn{2}{l|}{\tabhead{Grootzeil}}   & \multicolumn{2}{l|}{\tabhead{Lopend want}} & \multicolumn{2}{l|}{\tabhead{Casco}}  \\
	\textbf{\WIL1}		& Fok               & \textbf{24}      & Zeilteken              & \textbf{\WIL45}        & Fokkeschoot          & \textbf{\WIL68}     & Doft               \\
	\textbf{2}          & Tophoek           & \textbf{25}      & Zeilnummer             & \textbf{\WIL46}        & Grootschoot          & \textbf{69} & Dofthouder         \\
	\textbf{3}          & Halshoek          & \textbf{26}      & Zeillat                & \multicolumn{2}{l|}{\tabhead{Casco}}      & \textbf{\WIL70}     & Vlonder/Denning    \\
	\textbf{4}          & Schoothoek        & \textbf{27}  & Baan                   & \textbf{47}        & Dolboord             & \textbf{71}     & Spant              \\
	\textbf{5}          & Voorlijk          & \multicolumn{2}{l|}{\tabhead{Gaffel}}     & \textbf{48}        & Boeisel              & \multicolumn{2}{l|}{\tabhead{Zwaard}} \\
	\textbf{6}       	& Achterlijk        & \textbf{\WIL28}      & Gaffel                 & \textbf{49}        & Berghout             & \textbf{\WIL72}     & Zwaard             \\
	\textbf{7}       	& Onderlijk         & \textbf{29}      & Spruit/gaffeldraad     & \textbf{\WIL50}        & Boeg                 & \textbf{73}     & Zwaardbout         \\
	\textbf{8}       	& Leuver            & \textbf{30} & Strop                  & \textbf{52}        & Vlak                 & \textbf{74}     & Zwaardloper        \\
	\multicolumn{2}{|l|}{\tabhead{Mast}}     & \textbf{\WIL31}      & Klauw                  & \textbf{53}        & Scheg                & \textbf{75}     & Zwaardgreep        \\
	\textbf{\WIL9}          & Mast              & \textbf{\WIL32}      & Marllijn               & \textbf{54}        & Spiegel              & \textbf{\WIL76}     & Zwaardpen          \\
	\textbf{10}     & Windvaantje       & \multicolumn{2}{l|}{\tabhead{Giek}}       & \textbf{\WIL55}        & Voordek              & \textbf{\WIL77}     & Zwaardkast         \\
	\textbf{11}     & Mastring          & \textbf{\WIL33}      & Giek                   & \textbf{\WIL56}        & Achterdek            & \textbf{\WIL78}     & Zwaardplaatje      \\
	\textbf{\WIL12}         & Rijglijn          & \textbf{34}      & Lummelbeslag           & \textbf{57}        & Kim                  & \textbf{\WIL79}     & Mastkoker          \\
	\textbf{\WIL13}         & Mastbout          & \textbf{\WIL35}      & Grootschootring        & \textbf{\WIL58}        & Luchtkast            & \textbf{\WIL80}     & Kikker             \\
	\textbf{14}     & Grendelbout       & \textbf{\WIL36}      & Wervel                 & \textbf{\WIL59}        & Hanenkam             & \multicolumn{2}{l|}{\tabhead{Roer}}   \\
	\multicolumn{2}{|l|}{\tabhead{Grootzeil}}& \textbf{\WIL37}      & Pettenlijntje          & \textbf{\WIL60}        & Sleepoog             & \textbf{\WIL81}     & Helmstok           \\
	\textbf{\WIL15}         & Grootzeil         & \multicolumn{2}{l|}{\tabhead{Staand want}}& \textbf{61}        & Hijsoog              & \textbf{82}     & Roerkoning         \\
	\textbf{16}     & Tophoek           & \textbf{\WIL38}      & Voorstag               & \textbf{62}    & Leioog               & \textbf{83}     & Roerhaak           \\
	\textbf{17}     & Klauwhoek         & \textbf{\WIL39}      & Zijstag                & \textbf{63}    & Grootschootoog       & \textbf{\WIL84}     & Roerblad           \\
	\textbf{18}     & Halshoek          & \textbf{ 40} & Voorstagspanner        & \textbf{ 64}   & Landvastoog          & \textbf{85}     & Vingerling         \\
	\textbf{19}     & Schoothoek        & \multicolumn{2}{l|}{\tabhead{Lopend want}}& \textbf{65}    & Wrikgat              & \multicolumn{2}{l|}{\tabhead{Vlag}}   \\
	\textbf{20}     & Bovenlijk         & \textbf{\WIL41}      & Fokkeval               & \textbf{\WIL66}        & Dol                  & \textbf{86}  & Vlag               \\
	\textbf{\WIL21}         & Voorlijk          & \textbf{\WIL42}      & Klauwval               & \textbf{\WIL67}        & Dolpot               & \textbf{87}  & Vlaggenstok        \\
	\textbf{\WIL22}         & Achterlijk        & \textbf{\WIL43}      & Piekeval               & \textbf{}          &                      & \textbf{88}  & Knop               \\
	\textbf{23}     & Onderlijk         & \textbf{\WIL44}      & Kraanlijn / dirk       & \textbf{}          &                      & \textbf{}       &                    \\ \hline
\end{tabular}
	
	\setlength\extrarowheight{0pt} %Reset
	\renewcommand{\arraystretch}{1} %Reset
	
\end{table}

\section{Conclusie}
Naast dat je nu bekend bent met de bootonderdelen uit tabel \ref{table:vletwel}, zijn dit al de zeiltermen uit de vorige paragrafen die je kent en begrijpt.
\begin{itemize}[label=]
\begin{multicols}{4}
    \item Bakboord
    \item Stuurboord
    \item Loefzijde
	\item Lijzijde 
    \item Hoge kant
	\item Lage kant	   
    \item Hogerwal
    \item Lagerwal
    \item In de wind
    \item Aan de wind
    \item Halve wind
    \item Ruime wind
    \item Voor de wind
    \item Oploeven
    \item Afvallen
    \item Overstag gaan
    \item Opkruisen
    \item Gijpen

\end{multicols}
\end{itemize}
