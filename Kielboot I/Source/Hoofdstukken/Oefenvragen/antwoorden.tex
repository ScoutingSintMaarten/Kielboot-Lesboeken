\chapter{Antwoorden}
\vspace{-120px}
%%%% Hoofdstuk 1 %%%%%
\begin{table}[h]
	\centering
	\begin{tabular}{c|c|c|m{9.5cm}}
	\textbf{Hfd.}       & \textbf{Vraag} & \textbf{Antwoord} & \textbf{Toelichting}                                             \\ \hline
	\multirow{11}{*}{\sffamily\bfseries{\textcolor{ocre}{\LARGE1}} } & 1  & B & \\ \cline{2-4}          
	& 2 & B &  \\ \cline{2-4} 
	& 3 & A &  \\ \cline{2-4} 
	& 4 & D &  \\ \cline{2-4} 
	& 5 & A &  I: Hogerwal, II: Hoge kant, III: lage kant en IV: lagerwal\\ \cline{2-4} 
	& 6 & A &  \\ \cline{2-4} 
	& 7 & C &  A Aan de wind, B voor de wind, C ruime wind,   D halve wind\\ \cline{2-4} 
	& 8 & B &  A Ruime wind,  B aan de wind,  C voor de wind, D halve wind\\ \cline{2-4} 
	& 9 & D &  A Aan de wind, B ruime wind,   C halve wind,   D voor de wind\\ \cline{2-4} 
	& 10 & A &  \\ \cline{2-4} 
	& 11 & C & \\ \hline
	\multirow{4}{*}{\sffamily\bfseries{\textcolor{ocre}{\LARGE2}} } & 1   & B         &  \\ \cline{2-4} 
	& 2 & B &  \\ \cline{2-4} 
	& 3 & D &  \\ \cline{2-4} 
	& 4 & A & \\ \hline
	\multirow{6}{*}{\sffamily\bfseries{\textcolor{ocre}{\LARGE3}} } & 1   & A         & Zeilboot gaat voor spierkracht gaat voor motorboot \\ \cline{2-4} 
	& 2 & B &  Stuurboordswal gaat voor\\ \cline{2-4}  
	& 3 & A &  Zeilboten onderling: Zeilen over bakboord gaan voor\\ \cline{2-4} 
	& 4 & A &  Zeilboten onderling: Loef wijkt voor lij \\ \cline{2-4} 
	& 5 & C &  Zeilboot gaat voor spierkracht gaat voor motorboot \\ \cline{2-4} 
	& 6 & A &  Grote schepen gaan voor op kleine schepen\\ \hline 
	\multirow{6}{*}{\sffamily\bfseries{\textcolor{ocre}{\LARGE4}} } & 1 & A & Platteknoop voor gelijke dikte, schootsteek voor ongelijke dikte\\ \cline{2-4} 
	& 2 & B &  \\ \cline{2-4} 
	& 3 & D & Met een slipsteek kun je een mastworp `borgen' \\ \cline{2-4} 
	& 4 & D & A Mastworp, B Dubbele halve steek, C Slipsteek, D Halve steek\\ \cline{2-4} 
	& 5 & D &  \\ \cline{2-4} 
	& 6 & B &  \\ \hline 
	\multirow{5}{*}{\sffamily\bfseries{\textcolor{ocre}{\LARGE5}} } & 1 & B &  \\ \cline{2-4} 
	& 2 & B & Het zeil bolt tegen, dus het staat te los \\ \cline{2-4} 
	& 3 & B & Het fok valt af en het zeil loeft op \\ \cline{2-4} 
	& 4 & B & \\ \cline{2-4} 
	& 5 & D & 
	\end{tabular}
\end{table}

