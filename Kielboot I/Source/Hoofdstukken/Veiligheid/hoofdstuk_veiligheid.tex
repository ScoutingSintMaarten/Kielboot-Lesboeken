\header{3}
\chapter{Veiligheid}
\section{Inleiding}
Wanneer je wil gaan zeilen is het belangrijk dat dit veilig gebeurt. Hiervoor is het nodig om kennis te hebben over je reddingsvest en te weten wat je moet doen als je boot omslaat. Deze punten worden in dit hoofdstuk behandeld.
\section{Reddingsvest}
Een reddingsvest is een belangrijk onderdeel van de veiligheid aan boord. Er zijn vijf situaties waar je een reddingsvest aan moet:
\begin{enumerate}
\begin{multicols}{2}
    \item Als je boots/schipper het zegt
    \item Als de staf het zegt
    \item Als de waterpolitie het zegt 
    \item Als je het zelf wilt 
    \item Wanneer je een regenjas, regenbroek of kaplaarzen aan hebt
\end{multicols}
\end{enumerate}
Daarnaast zijn er een aantal strenge eisen aan reddingsvesten. Een reddingsvest moet:
\begin{itemize}
    \item Je snel op je rug draaien
    \item Je mond boven het water houden
    \item Handvatten hebben waar iemand mee uit het water getild kan worden
    \item Oranje of rood zijn.
\end{itemize}

In figuur \ref{pic:redvest} is een reddingsvest te zien met de belangrijkste onderdelen.

\begin{figure}[H]
	\centering
	\includegraphics[width=0.6\textwidth]{Hoofdstukken/Veiligheid/pdf/redvest.pdf}
	\caption{Reddingsvest}
	\centering
	\label{pic:redvest}
\end{figure}


\section{Omslaan}
Wanneer je boot is omgeslagen, \textbf{blijf je bij je boot}. Het is namelijk altijd gevaarlijker om te gaan zwemmen dan om bij je boot te blijven. Hier zijn een aantal redenen voor: ten eerste koel je veel minder snel af als je boven op je boot zit, of eraan hangt. Ook raak je zo minder vermoeid dan wanneer je zwemt. Daarnaast ben je makkelijker te vinden voor mensen die hulp willen bieden.

\section{Gedragsregels}
De belangrijkste en meest voorkomende gedragsregels zijn de volgende:
\begin{itemize}
    \item Houd de schippersgroet in ere
    \item Kom niet op andermans schip zonder toestemming
    \item Houd je schip en  omgeving schoon
\end{itemize}
\subsection*{Schippersgroet}
Op het water is het een gewoonte om als schippers (roergangers) onderling naar elkaar te zwaaien. Dit staat bekend als ``de schippersgroet''. Niet alleen is het een vorm van beleefdheid, maar je weet hierdoor ook zeker dat de schipper van het andere schip jou gezien heeft. 


\section{Conclusie}
Je hebt in dit hoofdstuk geleerd wat belangrijk is om veilig te zeilen. Zo zijn er regels voor wanneer je een reddingsvesten moet dragen en zijn er regels voor hoe je met elkaar omgaat op het water. Na het lezen van dit hoofdstuk ben je hier bekend mee.