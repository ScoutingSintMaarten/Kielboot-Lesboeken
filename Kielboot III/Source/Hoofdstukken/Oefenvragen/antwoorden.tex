\chapter{Antwoorden}
\vspace*{-120px}
\begin{table}[h!]
	\centering
	\begin{tabular}{c|c|c|m{9.5cm}}
		\textbf{Hfd.}       & \textbf{Vraag} & \textbf{Antwoord} & \textbf{Toelichting}                                             \\ \hline
		\multirow{10}{*}{\sffamily\bfseries{\textcolor{ocre}{\LARGE1}} } 
		& 1 & B & \\ \cline{2-4}          
		& 2 & A &  \\ \cline{2-4} 
		& 3 & A &  \\ \cline{2-4} 
		& 4 & D &  \\ \cline{2-4} 
		& 5 & A &  \\ \cline{2-4} 
		& 6 & A &  I: Hogerwal, II: Hoge kant, III: lage kant en IV: lager wal \\ \cline{2-4} 
		& 7 & D &  \\ \cline{2-4} 
		& 8 & C &  A aan de wind,  B voor de wind,  C ruime wind, D halve wind\\ \cline{2-4} 
		& 9 & B & \\ \cline{2-4} 
		& 10 & A &  \\ \cline{2-4} \hline
		\multirow{8}{*}{\sffamily\bfseries{\textcolor{ocre}{\LARGE2}} } 
		& 1 & D & Een reddingsvest moet je binnen \textit{15 seconden} op je rug draaien \\ \cline{2-4} 
		& 2 & C &  \\ \cline{2-4} 
		& 3 & A &  \\ \cline{2-4} 
		& 4 & A & \\ \cline{2-4} 
		& 5 & B & \\ \cline{2-4} 
		& 6 & D & \\ \cline{2-4} 
		& 7 & B & Krimpen is tegen de klok in, ruimen is met de klok mee\\ \cline{2-4} 
		& 8 & C & \\ \cline{2-4}  \hline
		\multirow{5}{*}{\sffamily\bfseries{\textcolor{ocre}{\LARGE3}} } 
		& 1 & A &  Bij B is tegenliggende \textit{niet} vaart mogelijk. Bij C is doorvaart verboden. Bij D is doorvaart aanstonds toegestaan.\\ \cline{2-4} 
		& 2 & A &  A gaat aanstonds open, B is al open en gaat sluiten, C is open en D is geeft geen informatie over openen\\ \cline{2-4} 
		& 3 & C &  Er zijn nog 8 streepjes zichtbaar van de 4 meter, dus 4,8m\\ \cline{2-4} 
		& 4 & C &  \\ \cline{2-4}
		& 5 & C &  \\ \cline{2-4} \hline
		\multirow{20}{*}{\sffamily\bfseries{\textcolor{ocre}{\LARGE4}} } 
		& 1 & A & Kruisende koers regel 4: Zeilboot gaat voor spierkracht gaat voor motorboot\\ \cline{2-4} 
		& 2 & A & Tegengestelde koers regel 4: Zeilen over bakboord gaan voor \\ \cline{2-4} 
		& 3 & B & Kruisende koers regel 6.2: Loef wijkt voor lij \\ \cline{2-4} 
		& 4 & A & Kruisende koers regel 6.1: Zeilen over bakboord gaan voor \\ \cline{2-4} 
		& 5 & C & Kruisende koers regel 4: Zeilboot gaat voor spierkracht gaat voor motorboot. Dus C-A-B \\ \cline{2-4} 
		& 6 & B & Kruisende koers regel 2: Grote schepen gaan voor op kleine schepen \\ \cline{2-4} 
		& 7 & B & Kruisende koers regel 3: Hoofd gaat voor nevenvaarwater \\ \cline{2-4} 
		& 8 & B & Tegengestelde koers regel 1: Stuurboordswal gaat voor \\ \cline{2-4} 
		& 9 & B & Bij oplopen, wijk de oploper. Het opgelopen schip kan indien nodig uitwijken\\ \cline{2-4} 
		& 10 & A & Goedzeemansschap: voorkom ten alle tijden een aanvaring bij gebrek aan regels \\ \cline{2-4} 
		& 11 & A & Kruisende koers regel 6.2: Loef wijkt voor lij \\ \cline{2-4} 
		& 12 & A & Bij het oversteken van een vaarwater heb je geen voorrang \\ \cline{2-4} 
		& 13 & B & Kruisende koers regel 1: Stuurboordswal gaat voor. B vaart stuurboordswal in de vaargeul\\ \cline{2-4} 
		& 14 & B & Kruisende koers regel 1: Stuurboordswal gaat voor \\ \cline{2-4} 
   		& 15 & A & Kruisende koers regel 2: Grote schepen gaan voor op kleine schepen \\ 
	\end{tabular}
\end{table}

\begin{table}[h]
	\centering
	\begin{tabular}{c|c|c|m{9.5cm}}
		\textbf{Hfd.}       & \textbf{Vraag} & \textbf{Antwoord} & \textbf{Toelichting}   
		                                          \\ \hline
   		\multirow{14}{*}{\sffamily\bfseries{\textcolor{ocre}{\LARGE4}} }
   		& 16 & A & Engte regel 1: Schip met stroom mee gaat voor \\ \cline{2-4} 
   		& 17 & A & Engte regel 5: Blokkade aan stuurboord verleent voorrang \\ \cline{2-4} 
   		& 18 & B & Engte regel 3: Spier gaat voor een motorschip  \\ \cline{2-4} 
   		& 19 & B & Engte regel 3: Zeilschip (bezeild) gaat voor motorschip \\ \cline{2-4} 
   		& 20 & B & Engte regel 3: Zeilschip (bezeild) gaat voor spier \\ \cline{2-4} 
   		& 21 & A & Engte regel 3: Spier gaat voor zeilschip (niet bezeild) \\ \cline{2-4} 
   		& 22 & A & Engte regel 1: Schip met stroom mee gaat voor \\ \cline{2-4} 
   		& 23 & A & Engte regel 2: Groot gaat voor klein \\ \cline{2-4} 
		& 24 & D & \\ \cline{2-4} 
		& 25 & D & \\ \cline{2-4} 
		& 26 & C & \\ \cline{2-4} 
		& 27 & D & \\ \cline{2-4} 
		& 28 & B & \\ \cline{2-4} 
		& 29 & A & \\ \cline{2-4} 
		& 30 & D & \\ \cline{2-4} \hline
   		\multirow{7}{*}{\sffamily\bfseries{\textcolor{ocre}{\LARGE5}} } 
   		& 1 & B & \\ \cline{2-4} 
   		& 2 & A &  \\ \cline{2-4} 
   		& 3 & C & Een groot motorschip op een tegengestelde koers met extra toplicht \\ \cline{2-4} 
   		& 4 & A & Een klein motorschip op een tegengestelde koers met een gecombineerde boordlichten op de boeg \\ \cline{2-4} 
   		& 5 & B &  Klein zeilschip groter dan 7 meter op een kruisende koers \\ \cline{2-4} 
   		& 6 & B &  Klein zeilschip met motor op een kruisende koers\\ \cline{2-4} 
   		& 7 & A &  Schip voortbewogen door zeil of spier kleiner dan 7 meter: enkel een rondom schijnend licht\\ \cline{2-4} \hline
   		\multirow{7}{*}{\sffamily\bfseries{\textcolor{ocre}{\LARGE6}} } 
   		& 1 & D & Platte knoop voor gelijke dikte, schootsteek voor ongelijke dikte \\ \cline{2-4} 
   		& 2 & D & Links: platte knoop, rechts: dubbele schootsteek \\ \cline{2-4} 
   		& 3 & C & A: Dubbele slippende halve steek, B Paal steek, C, Dubbele halve steek, D Halve steek \\ \cline{2-4} 
   		& 4 & B &  \\ \cline{2-4} 
   		& 5 & B &  Zie tabel \ref{table:touwwerk}\\ \cline{2-4} 
   		& 6 & A &  \\ \cline{2-4} \hline 
  		\multirow{8}{*}{\sffamily\bfseries{\textcolor{ocre}{\LARGE7}} } 
   		& 1 & A &  Lij om sneller op te loeven, loef om sneller af te vallen. Met
   		je zwaard op verlijer je. Je grootzeil helpt juist met oploeven,
   		vieren zal het niet sneller maken \\ \cline{2-4} 
   		& 2 & B &  De bolling aan de voorzijde geeft een te los zeil aan\\ \cline{2-4} 
   		& 3 & B &  Met de fok val je af, dus deze vier je. Met het grootzeil loef je
   		op dus deze trek je aan.\\ \cline{2-4} 
   		& 4 & B &  \\ \cline{2-4} 
   		& 5 & C &  De loefgierigheid  neemt af of verdwijnt zelfs en het schip wordt lijgierig  \\ \cline{2-4} 
   		& 6 & D &  Schepen met een kiel zijn gewichtsstabiel\\ \cline{2-4} 
   		& 7 & A &  \\ \cline{2-4}
   		& 8 & D &  \\ \cline{2-4} \hline
	\end{tabular}
\end{table}
