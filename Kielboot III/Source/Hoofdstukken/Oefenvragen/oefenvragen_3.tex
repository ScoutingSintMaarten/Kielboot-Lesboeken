\chapter{Bruggen \& Sluizen}
\vspace{-120px}
\question{1}{Bij welke brug is doorvaart toegestaan en is tegenliggende vaart mogelijk?}

\answerPicture{Hoofdstukken/Bruggen/pdf/brug_doorvaart_toegestaan.pdf}{Hoofdstukken/Bruggen/pdf/brug_doorvaart_geen_tegenligger.pdf}{Hoofdstukken/Bruggen/pdf/brug_doorvaart_verboden.pdf}{Hoofdstukken/Bruggen/pdf/brug_aanstonds_toegestaan.pdf}


\question{2}{Welke brug gaat bijna open?}

\answerPicture{Hoofdstukken/Bruggen/pdf/brug_aanstonds_toegestaan.pdf}{Hoofdstukken/Bruggen/pdf/brug_sluitend.pdf}{Hoofdstukken/Bruggen/pdf/brug_toegestaan.pdf}{Hoofdstukken/Bruggen/pdf/brug_doorvaart_toegestaan.pdf}

\question{3}{Wat is de doorvaarthoogte van de brug met deze hoogteschaal?}
\answerTextPicture{3,8 meter}{4,6 meter}{4,8 meter}{4,9 meter}{Hoofdstukken/Oefenvragen/pdf/hoogteschaal.pdf}

\question{4}{Wat betekent een dubbele gele ruit?}
\answerTextFour{Doorvaart verboden}{Doorvaart toegestaan, tegenliggers mogelijk}{Doorvaart toegestaan, tegenliggers niet mogelijk}{Brug buiten bediening}

\question{5}{De brug gaat al naar beneden. Mag je er nog onder door varen?}
\answerTextPicture{Ja, het groene licht brandt nog}{Ja, zolang het nog past}{Nee, \textit{tenzij} je niet meer kan stoppen}{Nee, je mag nooit onder een sluitende brug door varen}{Hoofdstukken/Bruggen/pdf/brug_sluitend.pdf}
