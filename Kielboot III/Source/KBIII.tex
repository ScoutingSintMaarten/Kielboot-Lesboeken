%%%%%%%%%%%%%%%%%%%%%%%%%%%%%%%%%%%%%%%%%
% The Legrand Orange Book
% LaTeX Template
% Version 2.3 (8/8/17)
%
% This template has been downloaded from:
% http://www.LaTeXTemplates.com
%
% Original author:
% Mathias Legrand (legrand.mathias@gmail.com) with modifications by:
% Vel (vel@latextemplates.com)
%
% License:
% CC BY-NC-SA 3.0 (http://creativecommons.org/licenses/by-nc-sa/3.0/)
%
% Compiling this template:
% This template uses biber for its bibliography and makeindex for its index.
% When you first open the template, compile it from the command line with the 
% commands below to make sure your LaTeX distribution is configured correctly:
%
% 1) pdflatex main
% 2) makeindex main.idx -s StyleInd.ist
% 3) biber main
% 4) pdflatex main x 2
%
% After this, when you wish to update the bibliography/index use the appropriate
% command above and make sure to compile with pdflatex several times 
% afterwards to propagate your changes to the document.
%
% This template also uses a number of packages which may need to be
% updated to the newest versions for the template to compile. It is strongly
% recommended you update your LaTeX distribution if you have any
% compilation errors.
%
% Important note:
% Chapter heading images should have a 2:1 width:height ratio,
% e.g. 920px width and 460px height.
%
%%%%%%%%%%%%%%%%%%%%%%%%%%%%%%%%%%%%%%%%%

%----------------------------------------------------------------------------------------
%	PACKAGES AND OTHER DOCUMENT CONFIGURATIONS
%----------------------------------------------------------------------------------------

\documentclass[11pt,fleqn]{book} % Default font size and left-justified equations
%----------------------------------------------------------------------------------------

\input{structure} % Insert the commands.tex file which contains the majority of the structure behind the template
\usepackage{graphicx}
\usepackage{lscape}
\usepackage{pgfgantt}
\usepackage{multicol}
\usepackage{caption} 
\usepackage{float}
\usepackage{pdfpages}
\usepackage{wrapfig}
\usepackage{enumitem}
\usepackage[export]{adjustbox}
\captionsetup[table]{skip=5pt}
\usepackage{parskip}  %Prevents having to use // after each paragraph
\usepackage{todonotes} %For easy Todo managment
\usepackage{array} %For defining custom colum types
\usepackage{multirow}

\newcolumntype{C}[1]{>{\centering\arraybackslash}p{#1}}
 
\pdfinclusioncopyfonts=1 % This fixes missing characher form figures
\pdfsuppresswarningpagegroup=1 % ignore page group warning

\hyphenation{verboden}
\hyphenation{water-beweging}
\hyphenation{voort-bewogen}
\hyphenation{geluids-signalen}



%Header and front folder selection

%%% Sint Maarten %%%
\newcommand{\header}[1]{\chapterimage{Banners/header_#1.png}}
\newcommand{\omslag}{Omslag/omslag.pdf}

\begin{document}

\setlength{\marginparwidth}{2cm}

%----------------------------------------------------------------------------------------
%	TITLE PAGE
%----------------------------------------------------------------------------------------

\begingroup
\thispagestyle{empty}
\begin{tikzpicture}[remember picture,overlay]
\node[inner sep=0pt] (background) at (current page.center) {\includegraphics[width=\paperwidth, page = 1]{\omslag}};
\end{tikzpicture}
\vfill
\endgroup

%----------------------------------------------------------------------------------------
%	COPYRIGHT PAGE
%----------------------------------------------------------------------------------------

\newpage
{\sffamily\bfseries Naam: }\rule{60mm}{.1pt}%
~\vfill
\thispagestyle{empty}
Voorpagina ontworpen door Christian Peppelman 2021. Achtergrond foto: Visit Aalsmeer. Website: \url{https://www.visitaalsmeer.nl/10x-weetjes-de-westeinderplassen/}. Tekening lelievlet gebaseerd op \textit{CWO Instructieboek} van de Katwijkse Zeeverkenners (zie dankwoord). 
Handelsmerken op het voorblad zijn van respectievelijke stichtingen of organisaties. De CWO en Scouting Nederland zijn niet betrokken geweest bij het opstellen van dit lesboek.

\textit{Opgeleverd op \today} % Printing/edition date


\header{0}
\chapter*{Voorwoord}

\section{Voorwoord}
Om ons huidige, inmiddels flink verouderde, kielboot theorieboek te vervangen ben ik begin 2018 begonnen met het schrijven van deze vernieuwde versie. Hoewel het oude boek zeker niet slecht was, was het duidelijk tijd voor wat vernieuwing. Qua inhoud is dit boek erg geïnspireerd op zijn voorloper, met de onderscheidende factor van duidelijkere afbeeldingen en illustraties, verbeterde teksten en een modern thema.

- Christian Peppelman
\section{Dankwoord}
Tijdens het opstellen van dit boek heb ik heel erg veel waardevolle feedback mogen ontvangen van medeleden van onze scoutingvereniging. Ik wil hen daar hartelijk voor bedanken! In het bijzonder Yara, Robert en Wouter voor de kritische, maar opbouwende feedback die dit boek gemaakt heeft tot wat het nu is.\\
Daarnaast wil ik ook graag de Katwijkse Zeeverkenners bedanken voor het online beschikbaar stellen van hun uitstekende lesboeken (\url{https://www.katwijksezeeverkenners.nl/cwo/instructieboeken/}). Het lesboek van de Katwijkse Zeeverkenners is een grote inspiratiebron geweest voor de figuren in dit lesboek.

\newpage
\section{Lesstof verantwoording}
De lesstof die in dit boek aan bod komt, is gemaakt om aan de eisen van de stichting Commissie Watersport Opleidingen (CWO) te voldoen voor de discipline kielboot II. Deze eisen zijn te vinden op \url{https://cwo.nl/leren-varen/kielboot}. Op sommige vlakken gaat dit boek uitgebreider in op de stof dan vanuit het CWO strikt noodzakelijk is. Hiervoor is gekozen omdat deze kennis een toegevoegde waarde kan bieden tijdens het zeilen op scouting.


\section{Document Informatie}
\subsection*{Licentie}
\begin{figure}[H]
	\centering
	\begin{minipage}[t]{0.60\textwidth}
		\vspace{-1.80cm}
		Dit boek is uitgebracht onder een Creative Commons
		'Naamsvermelding-NietCommercieel-GelijkDelen 4.0 Internationaal' (CC BY-NC-SA 4.0) licentie. Voor meer informatie: \url{https://creativecommons.org/licenses/by-nc-sa/4.0/}
	\end{minipage}
	\hfill
	\begin{minipage}[b]{0.35\textwidth}
	\includegraphics[width=\textwidth]{../Hoofdstukken/Informatie/CC-BY-NC-SA.png}
\end{minipage}
\end{figure}
\subsection*{Auteur informatie}
Dit boek is geschreven door Christian Peppelman.\\ 
Voor contact, vragen of verbeteringen kun je mailen naar: \href{mailto:cwo@sintmaartengroep.nl}{CWO@sintmaartengroep.nl} 
\subsection*{Gebruik}
Om optimaal gebruik te kunnen maken van dit lesboek, deze graag laten drukken in een geniete brochure in kleur. Gelieve het boek niet thuis te printen, inscannen of vermenigvuldigen in een manier die negatieve invloed op de kwaliteit heeft. Voor de originele bestanden of gedrukte varianten kun je contact opnemen of kijken op \url{https://sintmaartengroep.nl/}
\subsection*{Thema}
Het thema waar dit boek op gebaseerd is heet `The Legrand Orange Book' en is ontworpen door Mathias Legrand. Het thema is gedownload op \url{https://nl.overleaf.com/latex/templates/} en valt onder een Creative Commons BY-NC-SA 3.0 licentie.
\subsection*{Versiebeheer}
\begin{table}[H]
	\centering
	\begin{tabular}{c|l|p{8cm}}
		\textbf{Versie} & \textbf{Datum} & \textbf{Omschrijving} \\ \hline
		1.5 & 9 januari 2019 & Eerste druk  \\ \hline
	    1.6 & 23 augustus 2019 & Toevoeging Deel III: Zeilmanoeuvres  \\ \hline
		1.7 & 24 september 2019  & Spelling verbeteringen \\ \hline
		2.0 & 5 januari 2020  & Afronding versie 2 \\ \hline
		2.1 & 21 maart 2020  & Kleine verbeteringen \\ \hline
		2.2 & 23 maart 2021  & Figuur 2.5 vernieuwd en toevoeging antwoordenblad \\ \hline	
		2.3 & 6 november 2023  & Update hoofdstuk reglementen
	\end{tabular}
\end{table}


\textit{Versie 2.3 \hspace{1 cm} 6 november 2023}
%Druk verhoogt alleen met 0.x versie verhogingen of hoger

%----------------------------------------------------------------------------------------
%	TABLE OF CONTENTS
%----------------------------------------------------------------------------------------

%\usechapterimagefalse % If you don't want to include a chapter image, use this to toggle images off - it can be enabled later with \usechapterimagetrue

\chapterimage{Pictures/headers/header_1.png} % Table of contents heading image

\pagestyle{empty} % No headers

%\tableofcontents % Print the table of contents itself

%\cleardoublepage % Forces the first chapter to start on an odd page so it's on the right

\pagestyle{fancy} % Print headers again
\part{Theorie Lessen}
%\include{parts/lessen_indeling}
\header{2}
\chapter{Bootonderdelen \& Zeiltermen}
\section{Inleiding}
In dit hoofdstuk komen de verschillende onderdelen van de boot en een aantal zeiltermen aan bod. Deze termen en onderdelen zijn belangrijk om de volgende hoofdstukken in dit boek goed te begrijpen.

\section{Zeiltermen}
Voor duidelijke communicatie tijdens de les en in de boot is het van belang dat je een aantal zeiltermen kent. De belangrijkste termen worden hieronder besproken.

\subsection{Bakboord, Stuurboord, Loef en Lij}
Bakboord en stuurboord zijn het links en rechts \textbf{van de boot}, gezien vanaf het achterdek. Je moet altijd met de vaarrichting mee kijken. 

Loef en lij zeggen iets over de wind ten opzichte van je boot. De kant waar de wind de boot in komt, is de loefzijde, ook wel de hoge kant genoemd. De kant waar de wind de boot verlaat heet de lijzijde of lage kant. 
\begin{figure}[ht]
	\centering
	\includegraphics[width=0.9\textwidth]{Hoofdstukken/Onderdelen/pdf/wallen.pdf}
	\caption{Hoger- en Lagerwal}
	\centering
	\label{pic:hoog_laag}
\end{figure}

De hogerwal is de wal waar de wind vandaan komt. De lagerwal is de wal waar de wind naartoe waait. Al deze termen zijn te zien in figuur \ref{pic:hoog_laag}

\vfil\newpage

\subsection{Koersen}
Een koers vertelt iets over hoe je boot ligt ten opzichte van de wind. Alle koersen kun je zowel over bakboord, als stuurboord varen, behalve in de wind. Een overzicht van de koersen is te zien in figuur \ref{pic:koersen}. Wanneer je van koers verandert en naar de wind toe draait, loef je op. Wanneer je van de wind wegdraait heet dit afvallen. 

\begin{figure}[h]
	\centering
	\includegraphics[width=0.9\textwidth]{Hoofdstukken/Onderdelen/pdf/koersen.pdf}
	\caption{Windkoersen}
	\label{pic:koersen}
\end{figure}



\subsection{Boven- en benedenwinds}
Op het water kan je vaak op twee manieren ergens langs varen: bovenwinds en benedenwinds. Bovenwinds houdt in dat je ergens langs vaart aan de kant waar de wind ernaartoe blaast, de hoge kant van het object. Benedenwinds is het tegenovergestelde: dit is de kant waar de wind van het object weg blaast en dus de lage kant van het object. Deze termen zijn te zien in figuur \ref{pic:boven_benedenwinds}.
\begin{figure}[h]
  \centering
  \begin{minipage}[b]{0.7\textwidth}
   \centering
    \includegraphics[width=0.9\textwidth]{Hoofdstukken/Onderdelen/pdf/boven_en_benedenwinds.pdf}
    \caption{Boven- en benedenwinds passeren van een boei}
    \centering
    \label{pic:boven_benedenwinds}
  \end{minipage}
  \hfill
  \begin{minipage}[b]{0.29\textwidth}
    \centering
    \includegraphics[width=0.7\textwidth]{Hoofdstukken/Onderdelen/pdf/opkruisen.pdf}
    \caption{Opkruisen}
    \label{pic:opkruisen}
  \end{minipage}
\end{figure}



Tip: De termen boven- en benedenwinds kunnen goed van pas komen bij een zeilwedstrijd. Hier wordt vaak aangegeven of je een boei boven- of benedenwinds moet ronden.

\newpage
\subsection{Overig}
Hiernaast dien je ook bekend te zijn met de onderstaande termen:

\begin{itemize}
    \item \textit{Overstag}: Je gaat hier van aan de wind over de ene boeg naar aan de wind over de andere boeg. Bijvoorbeeld: van aan de wind over bakboord naar aan de wind over stuurboord.
    \item \textit{Gijpen}: Je gaat hier van voor de wind over de ene boeg naar voor de wind over de andere boeg. Bijvoorbeeld: van voor de wind over bakboord naar voor de wind over stuurboord.
    \item \textit{Opkruisen of laveren}: Hierbij vaar je tegen de wind in door steeds aan de wind te varen en dan overstag te gaan. Een voorbeeld hiervan is te zien in figuur \ref{pic:opkruisen}
    \item \textit{Killen van het zeil}: Je laat dan expres een deel van je zeil minder wind vangen. Dit doe je door je zeil te vieren totdat alleen het achterlijk nog wind vangt.
    \item \textit{Opschieten}: Wanneer je een lijn opschiet, rol je deze netjes op. Ook wel bekend als opbossen.
    \item \textit{Beleggen}: Als je een kikker belegt, leg je de lijn via een bepaalde knoop op de kikker. Deze knoop zal in het hoofdstuk `Schiemannen' behandeld worden.
    \item \textit{Bak}: Wanneer je je fok bak doet, zet je deze aan de hoge kant in plaats van de lage kant.
    \item \textit{Deinzen}: Dit is wanneer je achteruit dobbert met de neus van je boot in de wind. 
\end{itemize}


\newpage

\section{Bootonderdelen}
In figuur \ref{pic:vlet_nummers} is een tekening van een lelievlet te zien met maar liefst 88 gelabelde onderdelen. De namen van de onderdelen staan in tabel \ref{table:vletwel}. Alle onderdelen, behalve die met grijze nummers, moet je kennen.

\begin{figure}[h!]
	\centering
	\makebox[\textwidth][c]{\includegraphics[width=1.2\textwidth]{Hoofdstukken/Onderdelen/png/lelievlet_onderdelen.png}}
	\caption{Tekening lelievlet met nummers \protect\footnotemark}
	\centering
	\label{pic:vlet_nummers}
\end{figure}

\footnotetext{\textit{Lelievlet\_onderdelen.png}, https://www.willibrordusgroep.nl/Images/upload/cwo/lelievlet\_onderdelen.png, Feb 2021.
}



\begin{table}[h!]
	\centering
	\caption{Vletonderdelen}
	
	\setlength\extrarowheight{5pt} %Add height to center text vertically
	\renewcommand{\arraystretch}{0.75} %Shrink total heigt to keep row same heigt
	\newcommand{\tabhead}[1]{\cellcolor{ocre}{\color[HTML]{FFFFFF}\sffamily \textbf{#1}}}
	\newcommand{\NIL}[1]{\cellcolor{not}{#1}}
	\label{table:vletwel}
	
	\begin{tabular}{|ll|ll|ll|ll|}
	\multicolumn{2}{|l|}{\tabhead{Fok}}       & \multicolumn{2}{l|}{\tabhead{Grootzeil}}   & \multicolumn{2}{l|}{\tabhead{Lopend want}} & \multicolumn{2}{l|}{\tabhead{Casco}}  \\
	\textbf{\WIL1}		& Fok               & \textbf{24}      & Zeilteken              & \textbf{\WIL45}        & Fokkeschoot          & \textbf{\WIL68}     & Doft               \\
	\textbf{2}          & Tophoek           & \textbf{25}      & Zeilnummer             & \textbf{\WIL46}        & Grootschoot          & \textbf{69} & Dofthouder         \\
	\textbf{3}          & Halshoek          & \textbf{26}      & Zeillat                & \multicolumn{2}{l|}{\tabhead{Casco}}      & \textbf{\WIL70}     & Vlonder/Denning    \\
	\textbf{4}          & Schoothoek        & \textbf{27}  & Baan                   & \textbf{47}        & Dolboord             & \textbf{71}     & Spant              \\
	\textbf{5}          & Voorlijk          & \multicolumn{2}{l|}{\tabhead{Gaffel}}     & \textbf{48}        & Boeisel              & \multicolumn{2}{l|}{\tabhead{Zwaard}} \\
	\textbf{6}       	& Achterlijk        & \textbf{\WIL28}      & Gaffel                 & \textbf{49}        & Berghout             & \textbf{\WIL72}     & Zwaard             \\
	\textbf{7}       	& Onderlijk         & \textbf{29}      & Spruit/gaffeldraad     & \textbf{\WIL50}        & Boeg                 & \textbf{73}     & Zwaardbout         \\
	\textbf{8}       	& Leuver            & \textbf{30} & Strop                  & \textbf{52}        & Vlak                 & \textbf{74}     & Zwaardloper        \\
	\multicolumn{2}{|l|}{\tabhead{Mast}}     & \textbf{\WIL31}      & Klauw                  & \textbf{53}        & Scheg                & \textbf{75}     & Zwaardgreep        \\
	\textbf{\WIL9}          & Mast              & \textbf{\WIL32}      & Marllijn               & \textbf{54}        & Spiegel              & \textbf{\WIL76}     & Zwaardpen          \\
	\textbf{10}     & Windvaantje       & \multicolumn{2}{l|}{\tabhead{Giek}}       & \textbf{\WIL55}        & Voordek              & \textbf{\WIL77}     & Zwaardkast         \\
	\textbf{11}     & Mastring          & \textbf{\WIL33}      & Giek                   & \textbf{\WIL56}        & Achterdek            & \textbf{\WIL78}     & Zwaardplaatje      \\
	\textbf{\WIL12}         & Rijglijn          & \textbf{34}      & Lummelbeslag           & \textbf{57}        & Kim                  & \textbf{\WIL79}     & Mastkoker          \\
	\textbf{\WIL13}         & Mastbout          & \textbf{\WIL35}      & Grootschootring        & \textbf{\WIL58}        & Luchtkast            & \textbf{\WIL80}     & Kikker             \\
	\textbf{14}     & Grendelbout       & \textbf{\WIL36}      & Wervel                 & \textbf{\WIL59}        & Hanenkam             & \multicolumn{2}{l|}{\tabhead{Roer}}   \\
	\multicolumn{2}{|l|}{\tabhead{Grootzeil}}& \textbf{\WIL37}      & Pettenlijntje          & \textbf{\WIL60}        & Sleepoog             & \textbf{\WIL81}     & Helmstok           \\
	\textbf{\WIL15}         & Grootzeil         & \multicolumn{2}{l|}{\tabhead{Staand want}}& \textbf{61}        & Hijsoog              & \textbf{82}     & Roerkoning         \\
	\textbf{16}     & Tophoek           & \textbf{\WIL38}      & Voorstag               & \textbf{62}    & Leioog               & \textbf{83}     & Roerhaak           \\
	\textbf{17}     & Klauwhoek         & \textbf{\WIL39}      & Zijstag                & \textbf{63}    & Grootschootoog       & \textbf{\WIL84}     & Roerblad           \\
	\textbf{18}     & Halshoek          & \textbf{ 40} & Voorstagspanner        & \textbf{ 64}   & Landvastoog          & \textbf{85}     & Vingerling         \\
	\textbf{19}     & Schoothoek        & \multicolumn{2}{l|}{\tabhead{Lopend want}}& \textbf{65}    & Wrikgat              & \multicolumn{2}{l|}{\tabhead{Vlag}}   \\
	\textbf{20}     & Bovenlijk         & \textbf{\WIL41}      & Fokkeval               & \textbf{\WIL66}        & Dol                  & \textbf{86}  & Vlag               \\
	\textbf{\WIL21}         & Voorlijk          & \textbf{\WIL42}      & Klauwval               & \textbf{\WIL67}        & Dolpot               & \textbf{87}  & Vlaggenstok        \\
	\textbf{\WIL22}         & Achterlijk        & \textbf{\WIL43}      & Piekeval               & \textbf{}          &                      & \textbf{88}  & Knop               \\
	\textbf{23}     & Onderlijk         & \textbf{\WIL44}      & Kraanlijn / dirk       & \textbf{}          &                      & \textbf{}       &                    \\ \hline
\end{tabular}
	
	\setlength\extrarowheight{0pt} %Reset
	\renewcommand{\arraystretch}{1} %Reset
	
\end{table}

\section{Conclusie}
Naast dat je nu bekend bent met de bootonderdelen uit tabel \ref{table:vletwel}, zijn dit al de zeiltermen uit de vorige paragrafen die je kent en begrijpt.
\begin{itemize}[label=]
\begin{multicols}{4}
	\item Bakboord
	\item Stuurboord
 	\item Loefzijde
	\item Lijzijde
    \item Hoge kant
    \item Lage kant
    \item Hogerwal
   	\item Lagerwal
    \item In de wind
    \item Aan de wind
    \item Halve wind
    \item Ruime wind
    \item Voor de wind
    \item Oploeven
    \item Afvallen
    \item Bovenwinds
    \item Benedenwinds
    \item Overstag
    \item Gijpen
    \item Opkruisen
    \item Laveren
    \item Killen van het zeil 
    \item Opschieten
    \item Beleggen
    \item Bak
    \item Deinzen

\end{multicols}
\end{itemize}

\header{3}
\chapter{Veiligheid, Weer \& Vaarproblematiek}
\section{Inleiding}
Wanneer je wil gaan zeilen is het belangrijk dat dit veilig gebeurt. Om voor deze veiligheid te zorgen zijn een aantal punten van groot belang. Hierbij kan je denken aan een reddingsvest, kennis van het weer, kennis van je boot maar ook dat van andere boten. Al deze punten worden in dit hoofdstuk behandeld.
\section{Reddingsvest}
Een reddingsvest is een belangrijk onderdeel van de veiligheid aan boord. Er zijn 5 situaties waar je een reddingsvest aan moet:
\begin{enumerate}
\begin{multicols}{2}
    \item Als je boots het zegt
    \item Als de staf het zegt
    \item Als de waterpolitie het zegt 
    \item Als je het zelf wilt 
    \item Wanneer je een regenjas, regenbroek of kaplaarzen aan hebt
\end{multicols}
\end{enumerate}
Daarnaast zijn er een aantal strenge eisen aan reddingsvesten. Een reddingsvest moet:
\begin{itemize}
    \item Je binnen 15 seconden op je rug draaien
    \item Je mond 7 cm boven de het water houden
    \item De tekst \textit{``Front''} aan de voorkant bevatten
    \item In het Nederlands gegevens over het drijfvermogen en maximaal gewicht van de drager bevatten
    \item De naam en het adres van de fabrikant bevatten
    \item Voorzien zijn van handvatten waar iemand mee uit het water getild kan worden
    \item Oranje of rood zijn.
\end{itemize}

\section{Omslaan}
Wanneer je boot is omgeslagen, \textbf{blijf je bij je boot}. Het is namelijk altijd gevaarlijker om te gaan zwemmen dan om bij je boot te blijven. Hier zijn een aantal redenen voor: ten eerste koel je veel minder snel af als je boven op je boot zit, of eraan hangt. Ook raak je zo minder vermoeid dan wanneer je zwemt. Daarnaast ben je makkelijker te vinden voor mensen die hulp willen bieden.
\section{Gedragsregels}
De belangrijkste en meest voorkomende gedragsregels zijn de volgende:
\begin{itemize}
    \item Houd de schippersgroet in ere
    \item Kom niet op iemands anders schip zonder toestemming
    \item Houd je schip en  omgeving schoon
    \item Het is gebruikelijk om zeilwedstrijden voorrang te geven / te vermijden
\end{itemize}
\subsection*{Schippersgroet}
Op het water is het een gewoonte om als schippers (roergangers) onderling naar elkaar te zwaaien. Dit staat bekend als ``de schippersgroet''. Niet alleen is het een vorm van beleefdheid, maar je weet hierdoor ook zeker dat de schipper van het andere schip jou gezien heeft. 

\section{Weersinvloeden}
Wanneer je gaat varen is het weer van groot belang. Samen met het soort boot, het soort vaarwater en de kennis en ervaring van je bemanning kan dit bepalen of het wel veilig is om het water op te gaan. Een van de belangrijkste weerfactoren is de windkracht.

De kracht van de wind wordt vaak uitgedrukt in de windschaal van Beaufort. De schaal bevat 13 verschillende niveaus. In vroegere tijden werd de wind bepaald aan de hand van de effecten op de omgeving. Tegenwoordig is de schaal van Beaufort gebaseerd op snelheden in km/u. In tabel \ref{tab:beafort} is een overzicht van de verschillende niveaus met bijbehorende snelheid en het effect op de omgeving. Met je CWO Kielboot III mag je maar varen tot en met windkracht 5.

\begin{table}[h]
	\centering
	\caption{Windschaal Beaufort}
	\label{tab:beafort}
	\begin{tabular}{C{2cm}|c|C{1.5cm}|p{7cm}}
		\textbf{Windkracht {[}Bft.{]}} & \textbf{Omschrijving} & \textbf{Snelheid {[}km/u{]}} & \textbf{Effect\protect\footnotemark[1] }                                  \\ \hline
		0                              & Stil                  & 0 - 1                        & Rook stijgt recht of bijna recht omhoog          \\
		1                              & Zwak                  & 1 - 5                        & Windrichting goed af te leiden uit rookpluimen   \\
		2                              & Zwak                  & 6 - 11                       & Wind merkbaar in gezicht                         \\
		3                              & Matig                 & 12 - 19                      & Stof waait op                                    \\
		4                              & Matig                 & 20 - 28                      & Haar in de war, kleding flappert                 \\
		5                              & Vrij krachtig         & 29 - 38                      & Gekuifde golven op meren en kanalen              \\
		6                              & Krachtig              & 39 - 49                      & Paraplu's met moeite vast te houden              \\
		7                              & Hard                  & 50 - 61                      & Lastig tegen de wind in te lopen of fietsen      \\
		8                              & Stormachtig           & 62 - 74                      & Voortbewegen zeer moeilijk                       \\
		9                              & Storm                 & 75 - 88                      & Dakpannen waaien weg, kinderen waaien om         \\
		10                            & Zware storm            & 89 - 102                     & \noindent\parbox[c]{\hsize}{Grote schade aan gebouwen, volwassenen waaien om} \\
		11                             & Zeer zware storm      & 102 - 117                    & Enorme schade aan bossen                         \\
		12                             & Orkaan                & \textgreater 117             & Verwoestingen                                   
	\end{tabular}
\end{table}

\subsection{Weersomslag}
Tijdens het varen is het verstandig om goed te letten op een weersomslag. Een weersomslag betekent dat het weer heel snel verandert. Het zou dus heel hard kunnen gaan waaien, regenen of zelfs stormen. Er zijn een aantal kenmerken die dit aan kunnen geven. 
\begin{itemize}
    \begin{multicols}{2}
    \item Snel opkomende bewolking of wind
    \item Bloemkoolwolken
    \item Stilte voor de storm
    \item Plotselinge wind draaiing
    \end{multicols}
\end{itemize}
Ook zijn er nog twee belangrijke termen die met wind draaiing te maken hebben. Dit zijn: ruimen en krimpen. Wanneer de wind ruimt, draait deze met de richting van de wijzers van de klok mee. Een krimpende wind is een wind draaiing die tegen de richting van de klok in gaat. Krimpende wind wordt vaak geassocieerd met het verslechteren van het weer. Wanneer de wind dus sterk krimpt is het verstandig om het weer goed in de gaten te houden.
\footnotetext[1]{\textit{KNMI - Windschaal van Beaufort}, https://www.knmi.nl/kennis-en-datacentrum/uitleg/windschaal-van-beaufort, Okt 2022.
}

\newpage
\section{Vaarproblematiek andersoortige schepen}
\subsection{Dode hoek \hfill \textit{Figuur \ref{pic:dodehoek}}}
Net zoals in het verkeer bij vrachtauto's, kunnen grote schepen een dode hoek hebben. De dode hoek is het deel rondom het schip dat vanuit de stuurhut niet gezien kan worden. Sommige schepen hebben door hun vorm een dode hoek rondom het hele schip, niet alleen aan de voorkant. Als je bijvoorbeeld te dicht naast een schip vaart, kan de stuurman je mogelijk niet zien!

\subsection{Zuiging \hfill \textit{Figuur \ref{pic:zuiging}}}
Grote schepen hebben last van zuiging. De voorkant van het schip duwt het water weg en dit wordt aan de zij- en achterkant weer aangezogen. Kleine boten en zwemmers kunnen mee of onder gezogen worden. Blijf dus uit deze gebieden weg.

  \begin{center}
  \begin{minipage}[b]{0.32\textwidth}
    \begin{figure}[H]
 		\includegraphics[width=\textwidth]{Hoofdstukken/Veiligheid/pdf/dode_hoek.pdf}
        \caption{Dode hoek}
        \label{pic:dodehoek}
    \end{figure}
  \end{minipage}
    \hspace{2cm}
  \begin{minipage}[b]{0.32\textwidth}
  \begin{figure}[H]
 		\includegraphics[width=\textwidth]{Hoofdstukken/Veiligheid/pdf/zuiging.pdf}
        \caption{Zuiging}
        \label{pic:zuiging}
    \end{figure}
  \end{minipage}
  \end{center}


\subsection{Diepgang  \hfill \textit{Figuur \ref{pic:diepgang}}}
Veel wateren hebben een vaargeul, dit is een dieper deel van het vaarwater. Soms is dit aangegeven met boeien of tonnen. Grote boten die zwaar beladen zijn kunnen soms alleen in dit deel van het water varen. Ze zullen misschien niet kunnen wijken voor je en jij zal daar rekening mee moeten houden.

\subsection{Verlijeren  \hfill \textit{Figuur \ref{pic:verlijeren}}}
Net als bij een zeilboot, kunnen ook grote motorschepen verlijeren. Als de wind van de zijkant komt, zal deze het schip opzij duwen. Om dit te corrigeren zal hij een beetje schuin gaan varen. Dit zorgt ervoor dat hij meer ruimte inneemt en minder wendbaar is. Geef deze schepen de ruimte. 

  \begin{center}
  \begin{minipage}[b]{0.30\textwidth}
    \begin{figure}[H]
        \includegraphics[width=\textwidth]{Hoofdstukken/Veiligheid/pdf/diepgang.pdf}
        \caption{Diepgang}
        \label{pic:diepgang}
    \end{figure}
  \end{minipage}
    \hspace{2cm}
  \begin{minipage}[b]{0.25\textwidth}
  \begin{figure}[H]
        \includegraphics[width=\textwidth]{Hoofdstukken/Veiligheid/pdf/verlijeren.pdf}
        \caption{Verlijeren}
        \label{pic:verlijeren}
    \end{figure}
  \end{minipage}
  \end{center}


\section{Conclusie}
Je hebt in dit hoofdstuk geleerd wat belangrijk is om veilig te zeilen. Zo zijn er regels voor reddingsvesten, een gedragscode en is het slim om goed op het weer te letten - zowel voor als tijdens het varen. Als laatst is er nog gekeken naar vaarproblemen bij andere, voornamelijk grote, schepen.
\header{4}
\chapter{Bruggen \& Sluizen}
\section{Inleiding}
Bij langere tochten over het water zal je al snel te maken krijgen met bruggen en sluizen. In het BPR (Binnenvaartpolitiereglement) staan de regels voor het gebruik hiervan gedefinieerd. In dit hoofdstuk worden deze regels uitgelegd om zo veilig een brug of sluis te kunnen passeren.

\section{Vaste bruggen}
Bruggen zijn te onderscheiden in twee soorten: vaste en beweegbare bruggen. Een vaste brug, zoals in figuur \ref{pic:brug:vast}, kan niet open. Bij een beweegbare brug is een of meerdere wegdelen van de brug beweegbaar om grotere schepen te laten passeren. 

De brug in figuur \ref{pic:brug:vast} heeft drie vaste brugopeningen. De linker opening heeft een rood bord met een witte streep. Dit betekent dat doorvaart verboden is. De middelste opening heeft een enkele gele ruit. Dit betekent dat doorvaart toegestaan is, maar dat tegenliggende vaart mogelijk is. De rechter opening heeft twee gele ruiten. Dit betekent dat doorvaart is toegestaan en tegenliggende vaart verboden is. Aan de achterkant van deze vaaropening zal dan ook een `doorvaart verboden' bord hangen. 

Als je de keuze hebt tussen een of twee gele ruiten, maak dan altijd gebruik van de optie met de twee ruiten. Deze is het veiligst omdat je geen tegenliggers kunt hebben.
\begin{figure}[ht!]
  \centering
    \includegraphics[width=0.7\textwidth]{Hoofdstukken/Bruggen/pdf/brug_vast.pdf}
    \caption{}
    \label{pic:brug:vast}
\end{figure}

\section{Beweegbare bruggen}
Naast vaste bruggen zijn er ook beweegbare bruggen. Deze bruggen hebben lichten in plaats van borden. Vaak heeft een beweegbare brug naast een beweegbare opening, ook een vaste opening. Deze openingen beschikken dan ook over borden of lichten.  

\newpage

% --- Beweegbaar verboden door te varen
\begin{figure}[H]
  \centering
  \begin{minipage}[b]{0.18\textwidth}
    \includegraphics[width=\textwidth]{Hoofdstukken/Bruggen/pdf/brug_doorvaart_verboden.pdf}
    \caption{}
    \label{pic:brug:verboden}
  \end{minipage}
  \hfill
  \begin{minipage}[t]{0.75\textwidth}
  	\vspace{-2.5cm}
    Figuur \ref{pic:brug:verboden} betekent vrijwel hetzelfde als het rode bord uit figuur \ref{pic:brug:vast}. Doorvaart is verboden. Wanneer het echter de enige doorvaart is en je onder de gesloten brug past, mag je er wel door. Er kunnen dan ook tegenliggers aankomen.
  \end{minipage}
\end{figure}
% --- Beweegbaar doorvaart toegestaan, tegenliggende vaart mogelijk
\vspace{-0.75cm}
\begin{figure}[H]
	\centering
	\begin{minipage}[b]{0.18\textwidth}
		\includegraphics[width=\textwidth]{Hoofdstukken/Bruggen/pdf/brug_doorvaart_toegestaan.pdf}
		\caption{}
		\label{pic:brug:toegestaan}
	\end{minipage}
	\hfill
	\begin{minipage}[t]{0.75\textwidth}
	\vspace{-2.5cm}
	Figuur \ref{pic:brug:toegestaan} heeft dezelfde betekenis als een enkele gele ruit. De doorvaart is toegestaan, maar tegenliggende vaart is mogelijk. Wanneer je de optie hebt, kies dan voor de doorvaart met twee gele lichten. 
\end{minipage}
\end{figure}
% --- Beweegbaar doorvaart toegestaan, tegenliggende vaart niet mogelijk
\vspace{-0.75cm}
\begin{figure}[H]
\centering
\begin{minipage}[b]{0.18\textwidth}
	\includegraphics[width=\textwidth]{Hoofdstukken/Bruggen/pdf/brug_doorvaart_geen_tegenligger.pdf}
	\caption{}
	\label{pic:brug:toegestaan_tegenligger}
\end{minipage}
\hfill
\begin{minipage}[t]{0.75\textwidth}
	\vspace{-2.5cm}
	Figuur \ref{pic:brug:toegestaan_tegenligger} staat gelijk aan de twee gele ruiten. De doorvaart is toegestaan en tegenliggende vaart is niet mogelijk. Aan de andere kant van deze brug hangt een enkel rood licht of `verboden in te varen' bord.
\end{minipage}
\end{figure}
% --- Beweegbaar doorvaart aanstonds toegestaan
\vspace{-0.75cm}
\begin{figure}[H]
	\centering
	\begin{minipage}[b]{0.18\textwidth}
		\includegraphics[width=\textwidth]{Hoofdstukken/Bruggen/pdf/brug_aanstonds_toegestaan.pdf}
		\caption{}
		\label{pic:brug:aanstonds}
	\end{minipage}
	\hfill
	\begin{minipage}[t]{0.75\textwidth}
		\vspace{-2.5cm}
		Wanneer je niet onder een brug past en deze beweegbaar is, kan hij voor je opengaan. Wanneer een brug bijna opengaat, gaan de lichten branden als in figuur \ref{pic:brug:aanstonds}. Doorvaart is nog verboden totdat alleen het groene licht brandt
	\end{minipage}
\end{figure}
% --- Beweegbaar doorvaart toegestaan
\vspace{-0.75cm}
\begin{figure}[H]
	\centering
	\begin{minipage}[b]{0.18\textwidth}
		\includegraphics[width=\textwidth,]{Hoofdstukken/Bruggen/pdf/brug_toegestaan.pdf}
		\caption{}
		\label{pic:brug:vrij}
	\end{minipage}
	\hfill
	\begin{minipage}[t]{0.75\textwidth}
		\vspace{-2.5cm}
		Wanneer doorvaart door een beweegbare brug is toegestaan, brandt er een enkel groen licht zoals in figuur \ref{pic:brug:vrij}. Het kan ook zijn dat wanneer de brug open is, je eerst een enkel rood licht krijgt. Dit betekent dat de tegenliggers eerst mogen. Hierna zul jij een groen licht krijgen.
	\end{minipage}
\end{figure}
% --- Beweegbaar doorvaart aanstonds verboden
\vspace{-0.75cm}
\begin{figure}[H]
	\centering
	\begin{minipage}[b]{0.18\textwidth}
		\includegraphics[width=\textwidth]{Hoofdstukken/Bruggen/pdf/brug_sluitend.pdf}
		\caption{}
		\label{pic:brug:sluitend}
	\end{minipage}
	\hfill
	\begin{minipage}[t]{0.75\textwidth}
		\vspace{-2.5cm}
		Wanneer een brug bijna gaat sluiten of aan het sluiten is, gaat er een groen knipperend en rood licht branden, zoals in figuur \ref{pic:brug:sluitend}. De doorvaart is nu verboden, tenzij je redelijkerwijs niet meer kan stoppen. Dit is dus vergelijkbaar met een oranje verkeerslicht. 
	\end{minipage}
\end{figure}
% --- Beweegbaar buiten gebruik
\vspace{-0.75cm}
\begin{figure}[H]
	\centering
	\begin{minipage}[b]{0.18\textwidth}
		\includegraphics[width=\textwidth]{Hoofdstukken/Bruggen/pdf/brug_buiten_dienst.pdf}
		\caption{}
		\label{pic:brug:buiten}
	\end{minipage}
	\hfill
	\begin{minipage}[t]{0.75\textwidth}
		\vspace{-2.5cm}
		Als er een dubbel rood licht brandt (figuur \ref{pic:brug:buiten}), betekent het dat de brug buiten bediening is. De brugwachter kan dan bijvoorbeeld geen dienst hebben. Doorvaart is dan verboden. Wanneer er echter in het midden één of twee gele ruiten/lichten hangen gelden dezelfde regels als bij een enkel rood licht met gele ruit/licht.
	\end{minipage}
\end{figure}
% --- Beweegbaar gebied
\vspace{-0.75cm}
\begin{figure}[H]
	\centering
	\begin{minipage}[b]{0.18\textwidth}
		\includegraphics[width=\textwidth,]{Hoofdstukken/Bruggen/pdf/brug_aanbevolen_gebied.pdf}
		\caption{}
		\label{pic:brug:gebied}
	\end{minipage}
	\hfill
	\begin{minipage}[t]{0.75\textwidth}
		\vspace{-2.5cm}
		De ruiten in figuur \ref{pic:brug:gebied} geven iets aan over het aanbevolen vaargebied. Het is aanbevolen om binnen de groene ruiten te blijven varen. Dit kan te maken hebben met bijvoorbeeld een ondiepte of ander obstakel.
	\end{minipage}
\end{figure}
% --- Beweegbaar gebied verbod
\vspace{-0.75cm}
\begin{figure}[H]
\centering
\begin{minipage}[b]{0.18\textwidth}
	\includegraphics[width=\textwidth]{Hoofdstukken/Bruggen/pdf/brug_verboden_gebied.pdf}
	\caption{}
	\label{pic:brug:gebied_verbod}
\end{minipage}
\hfill
\begin{minipage}[t]{0.75\textwidth}
	\vspace{-2.5cm}
	Soms is het echter ook verboden om in bepaalde gebieden te varen. Dit wordt dan duidelijk gemaakt met de twee rode ruiten in figuur \ref{pic:brug:gebied_verbod}. Je moet dan tussen de rode ruiten in blijven en mag hier niet buiten varen.
\end{minipage}
\end{figure}

\section{Sluizen}
Een sluis wordt gebruikt om een boot te verplaatsen tussen twee wateren met een verschillende hoogte. Wanneer je een sluis in mag varen wordt, net als bij bruggen, bepaald door lichten.
Bij sluizen hebben de lichten vrijwel exact dezelfde betekenis als bij bruggen. Er zijn echter ook wat kleine verschillen. 

Vaak hangen er in een sluis zelf ook lichten. Deze maken duidelijk wanneer je de sluis uit mag varen. Wanneer er een sluiswachter aanwezig is moet je goed naar zijn instructies luisteren. Hij geeft vaak aan waar je moet gaan liggen in de sluis. 

% --- Sluis verbod
\hfill
\begin{figure}[H]
	\centering
	\begin{minipage}[b]{0.18\textwidth}
		\includegraphics[width=\textwidth]{Hoofdstukken/Bruggen/pdf/sluis_verboden.pdf}
		\caption{}
		\label{pic:sluis:verbod}
	\end{minipage}
	\hfill
	\begin{minipage}[t]{0.75\textwidth}
		\vspace{-2cm}
		Figuur \ref{pic:sluis:verbod} betekent net als bij bruggen dat doorvaart verboden is. Ook als de deuren helemaal open zijn, moet je wachten tot de lichten groen worden. Als er boten in de sluis liggen, moeten deze er namelijk eerst uit.
	\end{minipage}
\end{figure}
% --- Sluis verbod aanstonds
\vspace{-0.35cm}
\begin{figure}[H]
	\centering
	\begin{minipage}[b]{0.18\textwidth}	
		\includegraphics[width=\textwidth]{Hoofdstukken/Bruggen/pdf/sluis_aanstonds.pdf}
		\caption{}
		\label{pic:sluis:aanstonds}
	\end{minipage}
	\hfill
	\begin{minipage}[t]{0.75\textwidth}
		\vspace{-2cm}
		Wanneer de sluis bijna opengaat zullen de lichten aan gaan zoals in figuur \ref{pic:sluis:aanstonds}. Bij sommige sluizen is dit ook te zien als ze bijna gaan sluiten. Je mag er dan alleen nog in varen als je echt niet meer kan stoppen.
	\end{minipage}
\end{figure}
% --- Sluis toegestaan
\vspace{-0.35cm}
\begin{figure}[H]
	\centering
	\begin{minipage}[b]{0.18\textwidth}	
		\includegraphics[width=\textwidth]{Hoofdstukken/Bruggen/pdf/sluis_toegestaan.pdf}
		\caption{}
		\label{pic:sluis:toegestaan}
	\end{minipage}
	\hfill
	\begin{minipage}[t]{0.75\textwidth}
		\vspace{-2cm}
		Wanneer je de sluis in mag varen, geeft de sluis een enkel groen licht. Dit is te zien in figuur \ref{pic:sluis:toegestaan}. Wanneer de lichten groen zijn zullen alle boten die eerst in de sluis zaten, deze verlaten hebben. 
	\end{minipage}
\end{figure}
% --- Sluis buiten bedrijf
\vspace{-0.35cm}
\begin{figure}[H]
	\centering
	\begin{minipage}[b]{0.18\textwidth}	
		\includegraphics[width=\textwidth]{Hoofdstukken/Bruggen/pdf/sluis_buiten_dienst_dicht.pdf}
		\caption{}
		\label{pic:sluis:buiten}
	\end{minipage}
	\hfill
	\begin{minipage}[t]{0.75\textwidth}
		\vspace{-2cm}
		Een sluis kan net als een brug buiten bedrijf zijn. Dit wordt aangegeven met dubbele rode lichten uit figuur \ref{pic:sluis:buiten}. De deuren zullen in dit geval dicht zijn.
	\end{minipage}
\end{figure}
% --- Sluis buiten toegestaan
\vspace{-0.35cm}
\begin{figure}[H]
	\centering
	\begin{minipage}[b]{0.18\textwidth}	
		\includegraphics[width=\textwidth]{Hoofdstukken/Bruggen/pdf/sluis_buiten_dienst_open.pdf}
		\caption{}
		\label{pic:sluis:buiten_toegestaan}
	\end{minipage}
	\hfill
	\begin{minipage}[t]{0.75\textwidth}
		\vspace{-2cm}
		Het kan ook voorkomen dat de sluis buiten bedrijf is, maar doorvaart is toegestaan. Beide deuren staan dan open en de sluis geeft een dubbel groen licht, zie figuur \ref{pic:sluis:buiten_toegestaan}
	\end{minipage}
\end{figure}

\paragraph{Brug en sluis combinatie}
Het komt weleens voor dat er een sluis en brug direct naast elkaar geplaatst zijn. Let hierbij goed op de lichten. Het kan voorkomen dat je vrij lang voor de open brug moet wachten omdat de sluis eerst leeg moet varen. Wacht dan dus voor de brug, ook al pas je onder de brug door!

\section{Conclusie}
In dit hoofdstuk zijn alle lichten, tekens en regels voor bruggen en sluizen behandeld. Je weet nu wanneer het verboden en toegestaan is om een brug of sluis door te varen. Deze kennis is bijvoorbeeld heel erg van belang op een hike. Veel van de lichten hebben een logische betekenis en lijken soms zelfs een beetje op verkeerslichten. 

\header{5}
\chapter{Reglementen \& Voorrangsregels}
\section{Inleiding}
In dit hoofdstuk worden de regels en wetten op het water uitgelegd. Op de meeste wateren in Nederland wordt gebruik gemaakt van het Binnenvaartpolitiereglement, het BPR. Het BPR bevat alle regels over hoe je met elkaar om moet gaan op het water.

\section{Algemene reglementen}
Om het BPR goed te kunnen begrijpen, zullen er eerst een aantal algemene zaken besproken worden. Als eerste vijf definities van boten: motorschip, zeilschip, roeiboot, klein schip en groot schip. 

\begin{itemize}
    \item \textbf{Motorschip:} Een schip dat zich met een motor voortbeweegt.\hfill \raisebox{-0.4\height}{\includegraphics[height=1cm]{Hoofdstukken/Reglementen/pdf/motorboot.pdf}}
    \item \textbf{Zeilschip:} Een schip dat \textbf{alleen} zijn zeilen gebruikt om voort te bewegen.\hfill \raisebox{-0.4\height}{\includegraphics[height=1cm]{Hoofdstukken/Reglementen/pdf/zeilboot.pdf}}
    \item \textbf{Roeiboot:} Een schip dat \textbf{alleen} spierkracht gebruikt om zich voort te bewegen.\hfill \raisebox{-0.4\height}{\includegraphics[height=1cm]{Hoofdstukken/Reglementen/pdf/roeiboot.pdf}}
    \item \textbf{Klein schip:} Alle schepen onder de 20 meter.
    \item \textbf{Groot schip:} Schepen groter dan 20 meter.\hfill \raisebox{-0.4\height}{\includegraphics[height=1cm]{Hoofdstukken/Reglementen/pdf/groot_schip.pdf}}
\end{itemize}

\subsection{Goed zeemanschap}
Het goed zeemanschap is een hele belangrijke regel op het water. Deze regel houdt in dat de schipper er \textbf{alles} aan moet doen om een aanvaring te voorkomen. Dit betekent ook dat een schipper voor eigen veiligheid of die van anderen mag afwijken van de regels op het water.


\section{Voorrangsregels}
Voorrangssituaties zijn te verdelen in drie types: kruisende koersen, tegengestelde koeren en oplopende koersen. Deze drie koersen zijn te zien in figuur \ref{pic:voorrangkoers} en worden ook wel de voorrangskoersen genoemd. De koers die je vaart bepaalt met welke voorrangsregels je te maken hebt. 
\begin{figure}[H]
    \centering
    \includegraphics[width=1\textwidth]{Hoofdstukken/Reglementen/pdf/voorrangskoersen.pdf}
    \caption{Voorrangskoersen}
    \centering
    \label{pic:voorrangkoers}
\end{figure}
\vfil\newpage
De voorrangsregels hebben een volgorde. Je kijkt altijd eerst naar de bovenste regel. Als deze regel niet van toepassing is, ga je pas door naar de volgende. Dit doe je net zo lang totdat er een regel is die toe te passen is op jouw situatie.


\subsection{Kruisende Koersen}
Bij kruisende koersen kijk je naar de volgende voorrangsregels:
\begin{figure}[H]
	\centering
	\begin{minipage}[t]{0.70\textwidth}
	\textbf{1.} Het schip dat aan de stuurboordswal vaart heeft voorrang.\\
	\textit{Zeilschip A vaart aan de stuurboordswal en heeft voorrang op B}
	\end{minipage}
	\hfill
	\begin{minipage}[t]{0.20\textwidth}
		\raisebox{-0.5\height}{\includegraphics[width=\textwidth]{Hoofdstukken/Reglementen/pdf/kruis_stuurboordswal.pdf}}
		\label{pic:kr1}
	\end{minipage}
	\hfill
\end{figure}

\vspace{-0.7cm}

\begin{figure}[H]
	\centering
	\begin{minipage}[t]{0.70\textwidth}
	\textbf{2.} Grote schepen hebben voorrang op kleine schepen.\\
		\textit{Het grote motorschip A heeft voorrang op het kleine motorschip B}
	\end{minipage}
	\hfill
	\begin{minipage}[t]{0.20\textwidth}
	\raisebox{-0.5\height}{\includegraphics[width=\textwidth]{Hoofdstukken/Reglementen/pdf/kruis_groot_klein.pdf}}	
	\label{pic:kr2}
	\end{minipage}
	\hfill
\end{figure}

\vspace{-0.7cm}
\begin{figure}[H]
	\centering
	\begin{minipage}[t]{0.70\textwidth}
		\textbf{3.} Een zeilschip gaat voor een roeiboot gaat voor een motorschip.\\
		\textit{Zeilschip A heeft voorrang op roeiboot B en motorschip C\\
			 Roeiboot B heeft voorrang op motorschip C}
	\end{minipage}
	\hfill
	\begin{minipage}[t]{0.20\textwidth}
		\raisebox{-0.65\height}{\includegraphics[width=\textwidth]{Hoofdstukken/Reglementen/pdf/kruis_zsm.pdf}}	
		\label{pic:kr3}
	\end{minipage}
	\hfill
\end{figure}

\textbf{4.} Bij zeilschepen onderling zijn de volgende twee regels van belang:
\begin{figure}[H]
	\centering
	\hspace{0.02\textwidth}
	\begin{minipage}[t]{0.70\textwidth}
		\textbf{4.1}. Een zeilschip met zeilen over bakboord heeft voorrang.\\
		\textit{Zeilschip B (met zijn zeilen over bakboord) heeft voorrang op \\zeilschip A (met zijn zeilen over stuurboord)}
	\end{minipage}
	\hfill
	\begin{minipage}[t]{0.20\textwidth}
		\raisebox{-0.6\height}{\includegraphics[width=\textwidth]{Hoofdstukken/Reglementen/pdf/kruis_zeilboot_onderling_bakboord.pdf}}	
		\label{pic:kr41}
	\end{minipage}
	\hfill
\end{figure}

\vspace{-0.7cm}

\begin{figure}[H]
	\centering
	\hspace{0.02\textwidth}
	\begin{minipage}[t]{0.70\textwidth}
		\textbf{4.2.} Een zeilschip aan lij heeft voorrang op een zeilschip aan loef.\\
		\textit{Zeilschip A ligt aan loef van zeilschip B en verleent dus voorrang}
	\end{minipage}
	\hfill
	\begin{minipage}[t]{0.20\textwidth}
		\raisebox{-0.55\height}{\includegraphics[width=\textwidth]{Hoofdstukken/Reglementen/pdf/kruis_zeilboot_onderling_loef_lij.pdf}}	
		\label{pic:kr42}
	\end{minipage}
	\hfill
\end{figure}


\subsection{Tegengestelde Koersen}
Bij tegengestelde koersen kijk je naar de volgende voorrangsregels:
\begin{figure}[H]
	\centering
	\begin{minipage}[t]{0.70\textwidth}
		\textbf{1.} Het schip dat aan de stuurboordswal vaart heeft voorrang\\
		\textit{Zeilschip B vaart aan de stuurboordswal en heeft voorrang op A}
	\end{minipage}
	\hfill
	\begin{minipage}[t]{0.25\textwidth}
		\raisebox{-0.55\height}{\includegraphics[width=\textwidth]{Hoofdstukken/Reglementen/pdf/tegen_stuurboord.pdf}}
		\label{pic:tg1}
	\end{minipage}
	\hfill
\end{figure}

\begin{figure}[H]
	\centering
	\begin{minipage}[t]{0.70\textwidth}
		\textbf{2.} Grote schepen hebben voorrang op kleine schepen.\\
		\textit{Het grote motorschip B heeft voorrang op het kleine motorschip A}
	\end{minipage}
	\hfill
	\begin{minipage}[t]{0.25\textwidth}
		\raisebox{-0.55\height}{\includegraphics[width=\textwidth]{Hoofdstukken/Reglementen/pdf/tegen_groot_klein.pdf}}
		\label{pic:tg2}
	\end{minipage}
	\hfill
\end{figure}


\subsection{Oplopende koersen}
\begin{figure}[H]
	\centering
	\begin{minipage}[t]{0.70\textwidth}
		Bij het oplopen of inhalen, wijkt het oplopende schip. Het schip dat opgelopen wordt kan uitwijken als het nodig is.\\
		\textit{Zeilschip A (oploper) wijkt om op te lopen. Zeilschip B wijkt om wat extra ruimte te maken.}
	\end{minipage}
	\hfill
	\begin{minipage}[t]{0.25\textwidth}
		\raisebox{-0.8\height}{\includegraphics[width=\textwidth]{Hoofdstukken/Reglementen/pdf/oplopen.pdf}}
		\label{pic:op}
	\end{minipage}
	\hfill
\end{figure}

\subsection{Toevoegingen}
Bij zeilboten onderling is het belangrijk dat je, als het kan, altijd aan de loefzijde inhaalt. In figuur \ref{pic:oplopen_lij} zie je dat zeilschip B ingehaald wordt door zeilschip A aan de loefzijde. Hiermee neemt zeilschip A de wind uit de zeilen van zeilschip B. Het oplopen gaat hierdoor sneller.

Bij het oplopen aan de lijzijde vangt de oploper juist minder wind. Dit is te zien bij zeilschip B in figuur \ref{pic:oplopen_lij}. Hierdoor gaat het oplopen minder snel. Soms lukt het zelfs helemaal niet om op te lopen aan de lijzijde.

\begin{figure}[h]
	\centering
	\begin{minipage}[b]{0.49\textwidth}
		\centering
		\includegraphics[width=0.65\textwidth]{Hoofdstukken/Reglementen/pdf/oplopen_loef.pdf}
		\caption{Oplopen aan loef}
		\centering
		\label{pic:oplopen_loef}
	\end{minipage}
	\hfill
	\begin{minipage}[b]{0.49\textwidth}
		\centering
		\includegraphics[width=0.65\textwidth]{Hoofdstukken/Reglementen/pdf/oplopen_lij.pdf}
		\caption{Oplopen aan lij}
		\label{pic:oplopen_lij}
	\end{minipage}
\end{figure}
	
		
Daarnaast is het belangrijk te onthouden dat wanneer je vaarwater oversteekt, je geen voorrang hebt. Andere schepen moeten hun koers en snelheid niet of nauwelijks hoeven aan te passen voor jouw manoeuvre. 

\subsection{Voorrangsregels op een rij}
Om de voorrangsregels makkelijk te kunnen onthouden staan ze hieronder samengevat:\\\\
Bij \textbf{kruisende koersen} kijk je naar de volgende regels:
\vspace*{-0.15cm}
\begin{enumerate}
	\item Het stuurboordswal varende schip gaat voor
	\item Grote schepen gaan voor op kleine schepen
	\item Zeilboot gaat voor roeiboot gaat voor motorboot
	\item Zeilboten onderling: 
	
	\begin{enumerate}
		\item [4.1.]Zeilen over bakboord gaat voor
		\item [4.2.]Loef wijkt voor lij
	\end{enumerate}
\end{enumerate}

Bij \textbf{tegengestelde koersen} kijk je naar de volgende regels:
\vspace*{-0.15cm}
\begin{enumerate}
	\item Het stuurboordswal varende schip gaat voor
	\item Grote schepen gaan voor op kleine schepen
\end{enumerate}

Bij \textbf{oplopende koersen} wijkt de oploper. Het opgelopen schip kan indien nodig uitwijken.


\section{Conclusie}
Na het lezen van dit hoofdstuk heb je kennis van de voorrangsregels op het water. Een van de belangrijkste is het goed zeemanschap, wat inhoudt dat je alles doet om een gevaarlijke situatie of aanvaring te voorkomen. Daarnaast ken je de verschillende voorrangssituaties en volgorde van de voorrangsregels en weet je hoe je deze moet toepassen. 
\header{6}
\chapter{Optische Tekens}
\section{Inleiding}
Vanuit het BPR zijn er een aantal regels voor het dragen van optische tekens op je schip. Onder optische tekens vallen zowel navigatieverlichting als dagtekens. In dit hoofdstuk wordt uitgelegd wanneer schepen optische tekens moeten dragen en welke tekens dit zijn. Daarnaast leer je andere schepen aan hun tekens te identificeren.
\newcommand{\RemoveLine}{\vspace*{-2mm}}

\section{Lichten}
In het BPR staat voor verschillende scheepssoorten gespecificeerd welke navigatieverlichting deze moeten dragen. Het BPR verplicht schepen deze tekens te dragen tijdens de nacht (tijd tussen zonsondergang en zonsopgang).

\vspace*{-0.5cm}
\begin{figure}[H]
	\centering
	\begin{minipage}[t]{0.43\textwidth}
		Het BPR kent vier verschillende soorten lichten: toplicht, boordlichten, heklicht en rondom schijnend licht. In figuur \ref{pic:optische_tekens} is een overzicht van de verschillende lichten te zien en welke hoek deze beslaan. In tabel \ref{tab:licht_legenda} zijn de schematische weergaven van deze lichten te zien.
	\end{minipage}
	\hfill
	\begin{minipage}[t]{0.55\textwidth}
		\raisebox{-0.8\height}{\includegraphics[width=\textwidth]{Hoofdstukken/Optisch/pdf/lichtroos.pdf}}
		\RemoveLine
		\caption{}
		\label{pic:optische_tekens}
	\end{minipage}
\end{figure}

\begin{table}[h]
	\centering
	\caption{Legenda lichten}
	\label{tab:licht_legenda}
	\begin{tabular}{c@{}m{4cm}c@{}m{2.5cm}c@{}m{1.5cm}c@{}m{1.5cm}}
		\raisebox{-0.35\height}{\includegraphics[height=0.8cm]{Hoofdstukken/Optisch/pdf/rondom_schijnend_licht.pdf}} & \hspace*{1mm}Rondom schijnend licht
		&\raisebox{-0.35\height}{\includegraphics[height=0.8cm]{Hoofdstukken/Optisch/pdf/boordlicht_bakboord.pdf}}
		\raisebox{-0.35\height}{\includegraphics[height=0.8cm]{Hoofdstukken/Optisch/pdf/boordlicht_stuurboord.pdf}} & \hspace*{1mm} Boordlichten	
		&\scalebox{-1}[1]{\raisebox{-0.35\height}{\includegraphics[height=0.8cm]{Hoofdstukken/Optisch/pdf/heklicht.pdf}}} & \hspace*{1mm}Toplicht
		&\raisebox{-0.35\height}{\includegraphics[height=0.8cm]{Hoofdstukken/Optisch/pdf/heklicht.pdf}} &\hspace*{1mm} Heklicht
	\end{tabular}
\end{table}
%
\section{Dagtekens}
Door het dragen van dagtekens worden omliggende schepen op bepaalde zaken geattendeerd. De voornaamste gebruikte dagtekens zijn:

\begin{itemize}
	\item \textbf{Zwarte kegel:} De zwarte kegel wordt gebruikt om aan te geven dat een zeilschip, naast zijn zeilen, ook zijn motor gebruikt om voort te bewegen.
	\item \textbf{Zwarte bol:} Een zwarte bol geeft aan dat een schip voor anker ligt. 
	\item \textbf{Gele ruit:} Passagiersschepen onder de 20 meter dragen een gele ruit om aan te geven dat zij een passagiersschip zijn (meer dan 12 personen).
	\item \textbf{Sleep cilinder:} Een `sleep'-cilinder (wit-zwart-geel-zwart-wit) toont dat een schip aan het slepen is.
\end{itemize}


\vfil\newpage
\section{Tekens van schepen}
Verschillende schepen zijn verplicht verschillende optische tekens te dragen. Aan de hand van beschrijvingen en figuren worden deze verplichtingen in deze paragraaf toegelicht. 
\subsection{Grote schepen}
% --- Groot motorschip normaal
\begin{figure}[H]
	\centering
	\begin{minipage}[t]{0.75\textwidth}
	%\vspace{-3.5cm}
	\paragraph{Groot motorschip}
	Een alleenvarend groot motorschip voert 's nachts een toplicht (minimaal 4 meter hoog), boordlichten en een heklicht. Deze verlichting is te zien in figuur \ref{pic:optisch:grootmotor}. Het tweede toplicht (op de kajuit) is enkel verplicht als een schip langer is dan 110 meter en moet hoger zijn dat het voorste toplicht.
	\end{minipage}
	\hfill
	\begin{minipage}[t]{0.22\textwidth}
	\raisebox{-0.9\height}{\includegraphics[width=\textwidth]{Hoofdstukken/Optisch/pdf/groot_motorschip.pdf}
	}
	\RemoveLine
	\caption{}
	\label{pic:optisch:grootmotor}
\end{minipage}
\end{figure}
\vspace{-0.6cm}
% --- Groot zeilschip normaal
\begin{figure}[H]
	\centering
	\begin{minipage}[t]{0.75\textwidth}
		\paragraph{Groot zeilschip}
		Figuur \ref{pic:optisch:grootzeil} toont een grootzeilschip bij nacht. Het zeilschip draagt dan boordlichten, een heklicht en twee rondom schijnende lichten in de mast. Het bovenste rondom schijnende licht dient rood te zijn en de onderste groen. 
	\end{minipage}
	\hfill
	\begin{minipage}[t]{0.22\textwidth}
		\raisebox{-0.9\height}{\includegraphics[width=\textwidth]{Hoofdstukken/Optisch/pdf/groot_zeilschip.pdf}}
		\RemoveLine
		\caption{}
		\label{pic:optisch:grootzeil}
	\end{minipage}
\end{figure}
\vspace{-0.6cm}
% --- Groot schip stil
\begin{figure}[H]
	\centering
	\begin{minipage}[t]{0.75\textwidth}
		\paragraph{Stilliggen (oever)}
		Wanneer een groot schip 's nachts aan een oever is afgemeerd, dient deze een wit rondom schijnend licht te dragen. Dit licht moet op minstens 3 meter hoogte geplaatst zijn aan de zijde van het vaarwater. Deze situatie is te zien in figuur \ref{pic:optisch:groot_stil}.
	\end{minipage}
	\hfill
	\begin{minipage}[t]{0.22\textwidth}
		\raisebox{-0.9\height}{\includegraphics[width=\textwidth]{Hoofdstukken/Optisch/pdf/groot_motorschip_afmeren.pdf}}
		\RemoveLine
		\caption{}
		\label{pic:optisch:groot_stil}
	\end{minipage}
\end{figure}
\vspace{-0.6cm}
% --- Groot schip ankeren
\begin{figure}[H]
	\centering
	\begin{minipage}[t]{0.50\textwidth}
		\paragraph{Ankeren}
		Een groot schip wat 's nachts voor anker ligt moet dit kenbaar maken door het dragen van twee witte rondom schijnende toplichten: een licht voorop het schip en een lager licht achterop het schip. Dit is te zien figuur \ref{pic:optisch:groot_anker}.		
	\end{minipage}
	\hfill
	\begin{minipage}[t]{0.22\textwidth}
		\raisebox{-0.9\height}{\includegraphics[width=\textwidth]{Hoofdstukken/Optisch/pdf/groot_motorschip_ankeren}}
		\RemoveLine
		\caption{}
		\label{pic:optisch:groot_anker}
	\end{minipage}
	\hfill
	\begin{minipage}[t]{0.22\textwidth}
	\raisebox{-0.9\height}{\includegraphics[width=\textwidth]{Hoofdstukken/Optisch/pdf/groot_motorschip_ankeren_dag.pdf}}
	\RemoveLine
	\caption{}
	\label{pic:optisch:groot_anker2}
	\end{minipage}
\end{figure}
\vspace{-0.6cm}
Overdag draagt een groot schip wat geankerd is een zwarte bol. Deze bol dient opgehangen te worden op het voorschip. Deze situatie is te zien in figuur \ref{pic:optisch:groot_anker2}.

% --- Groot schip slepen
\begin{figure}[H]
	\centering
	\begin{minipage}[t]{0.50\textwidth}
		\paragraph{Slepen}
		Wanneer een groot motorschip 's nachts andere schepen sleept dient deze dat kenbaar te maken met de lichten die te zien zijn in figuur \ref{pic:optisch:groot_sleep}: twee toplichten loodrecht boven elkaar, twee boordlichten en een geel heklicht.
	\end{minipage}
	\hfill
	\begin{minipage}[t]{0.22\textwidth}
		\raisebox{-0.9\height}{\includegraphics[width=\textwidth]{Hoofdstukken/Optisch/pdf/groot_motorschip_sleep.pdf}}
		\RemoveLine
		\caption{}
		\label{pic:optisch:groot_sleep}
	\end{minipage}
	\hfill
	\begin{minipage}[t]{0.22\textwidth}
		\raisebox{-0.9\height}{\includegraphics[width=\textwidth]{Hoofdstukken/Optisch/pdf/groot_motorschip_sleep_dag.pdf}}
		\RemoveLine
		\caption{}
		\label{pic:optisch:groot_sleep2}
	\end{minipage}
\end{figure}
\vspace{-0.6cm}
Ieder opvolgend gesleept schip dient een wit rondom schijnend licht te dragen. Het laatste schip dient ook een heklicht te dragen.

Overdag draagt het slepende schip een wit-zwart-geel gekleurde cilinder. Ieder opvolgend schip draagt een gele bol. Het slepende schip is te zien in figuur \ref{pic:optisch:groot_sleep2}.


\subsection{Kleine schepen}

% --- Kleine motorschip
\begin{figure}[H]
	\centering
	\begin{minipage}[t]{0.50\textwidth}
		\paragraph{Klein motorschip}
		's Nachts draagt een klein motorschip navigatieverlichting bestaande uit een toplicht, twee boordlichten en een heklicht. Deze opstelling is te zien in figuur \ref{pic:optisch:klein_motor}. Op deze configuratie zijn echter een aantal variaties toegestaan.		
	\end{minipage}
	\hfill
	\begin{minipage}[t]{0.22\textwidth}
		\raisebox{-0.9\height}{\includegraphics[width=\textwidth]{Hoofdstukken/Optisch/pdf/klein_motorschip.pdf}}
		\RemoveLine
		\caption{}
		\label{pic:optisch:klein_motor}
	\end{minipage}
	\hfill
	\begin{minipage}[t]{0.22\textwidth}
		\raisebox{-0.9\height}{\includegraphics[width=\textwidth]{Hoofdstukken/Optisch/pdf/klein_motorschip_alt.pdf}}
		\RemoveLine
		\caption{}
		\label{pic:optisch:klein_motor2}
	\end{minipage}
\end{figure}
\vspace{-0.6cm}
Zo is het toegestaan de boordlichten in de boeg van het schip te dragen. Ook kan het hek- en toplicht samengevoegd worden in één enkel rondomschijnend toplicht wat minstens een meter hoger is dan de boordlichten. Deze beide variaties zijn in figuur \ref{pic:optisch:klein_motor2} afgebeeld.


% --- Kleine openmotorschip
\begin{figure}[H]
	\centering
	\begin{minipage}[t]{0.75\textwidth}
		\paragraph{Klein open motorschip}
		Een klein \textit{open} motorschip met een lengte van minder dan 7 meter een maximale snelheid van minder dan 13 km per uur hoeft enkel en wit rondom schijnend licht te voeren. Dit is te zien in figuur \ref{pic:optisch:klein_openmotor}.
	\end{minipage}
	\hfill
	\begin{minipage}[t]{0.22\textwidth}
		\raisebox{-0.9\height}{\includegraphics[width=\textwidth]{Hoofdstukken/Optisch/pdf/klein_open_motorschip.pdf}}
		\RemoveLine
		\caption{}
		\label{pic:optisch:klein_openmotor}
	\end{minipage}
\end{figure}


% --- Kleine zeilschip groter dan 7 meter
\begin{figure}[H]
	\centering
	\begin{minipage}[t]{0.50\textwidth}
		\paragraph{Klein zeilschip (>7 meter)}
		Een klein zeilschip draagt 's nachts wanneer deze enkel zeilt drie lichten: twee boordlichten en een heklicht. Deze mogen in twee uitvoeringen gedragen worden. De eerste optie, figuur \ref{pic:optisch:klein_zeil}, heeft boordlichten nabij de boeg van het schip en een heklicht achter op.
	\end{minipage}
	\hfill
	\begin{minipage}[t]{0.22\textwidth}
		\raisebox{-0.9\height}{\includegraphics[width=\textwidth]{Hoofdstukken/Optisch/pdf/klein_zeilschip.pdf}}
		\RemoveLine
		\caption{}
		\label{pic:optisch:klein_zeil}
	\end{minipage}
	\hfill
	\begin{minipage}[t]{0.22\textwidth}
		\raisebox{-0.9\height}{\includegraphics[width=\textwidth]{Hoofdstukken/Optisch/pdf/klein_zeilschip_alt.pdf}}
		\RemoveLine
		\caption{}
		\label{pic:optisch:klein_zeil2}
	\end{minipage}
\end{figure}%
\vspace{-0.6cm}%
De tweede optie verenigt de drie lichten in de mast. In figuur \ref{pic:optisch:klein_zeil2} is dit gecombineerde licht te zien.



% --- Kleine zeilschip met motor
\begin{figure}[H]
	\centering
	\begin{minipage}[t]{0.75\textwidth}
		\paragraph{Klein zeilschip met motor}
		Wanneer een klein zeilschip 's nachts naast zijn zeilen ook een motor gebruikt om zich voort te bewegen, dient deze een toplicht te dragen bij de al voor het zeilschip verplichte lichten. Dit is afgebeeld in figuur \ref{pic:optisch:klein_zeil_motor}. Overdag draagt een zeilschip in deze situatie een kegel. Dit wordt verder toegelicht in paragraaf \ref{par:optisch:overig}
	\end{minipage}
	\hfill
	\begin{minipage}[t]{0.22\textwidth}
		\raisebox{-0.9\height}{\includegraphics[width=\textwidth]{Hoofdstukken/Optisch/pdf/klein_zeilschip_motor.pdf}}
		\RemoveLine
		\caption{}
		\label{pic:optisch:klein_zeil_motor}
	\end{minipage}	
\end{figure}

% --- Kleine zeilschip >7 meter
\begin{figure}[H]
	\centering
	\begin{minipage}[t]{0.50\textwidth}
		\paragraph{Klein zeilschip (<7 meter) en spier}
		Een zeilschip kleiner dan 7 meter en kleine schepen voortbewogen door spierkracht hoeven 's nachts enkel een rondom schijnend licht te dragen wat van alle zijde goed zichtbaar is. Dit is weergegeven in figuur  \ref{pic:optisch:klein_zeil_klein} en \ref{pic:optisch:klein_spier} .
	\end{minipage}
	\hfill
	\begin{minipage}[t]{0.22\textwidth}
		\raisebox{-0.9\height}{\includegraphics[width=\textwidth]{Hoofdstukken/Optisch/pdf/klein_zeilschip_7m.pdf}}
		\RemoveLine
		\caption{}
		\label{pic:optisch:klein_zeil_klein}
	\end{minipage}
	\hfill
	\begin{minipage}[t]{0.22\textwidth}
		\raisebox{-0.9\height}{\includegraphics[width=\textwidth]{Hoofdstukken/Optisch/pdf/klein_spier.pdf}}
		\RemoveLine
		\caption{}
		\label{pic:optisch:klein_spier}
	\end{minipage}
\end{figure}%

% --- Kleine slepen
\begin{figure}[H]
	\centering
	\begin{minipage}[t]{0.75\textwidth}
		\paragraph{Klein gesleept schip}
		Wanneer een klein schip 's nachts gesleept wordt door een motorschip, dient deze een wit rondom schijnend licht te dragen. Dit is te zien in figuur \ref{pic:optisch:klein_sleep}. Overdag hoeven kleine schepen die gesleept worden geen speciale tekens te dragen.
	\end{minipage}
	\hfill
	\begin{minipage}[t]{0.22\textwidth}
		\raisebox{-0.9\height}{\includegraphics[width=\textwidth]{Hoofdstukken/Optisch/pdf/klein_gesleept.pdf}}
		\RemoveLine
		\caption{}
		\label{pic:optisch:klein_sleep}
	\end{minipage}
\end{figure}

% --- Kleine schip ankeren
\begin{figure}[H]
	\centering
	\begin{minipage}[t]{0.50\textwidth}
		\paragraph{Ankeren}
		's Nachts draagt een klein geankerd schip, net als een groot schip, één wit rondom schijnend toplicht zoals te zien is in figuur \ref{pic:optisch:klein_anker}. Deze verplichting vervalt echter als de omgeving goed verlicht is of het schip op een veilige plaats buiten het vaarwater ligt.
	\end{minipage}
	\hfill
	\begin{minipage}[t]{0.22\textwidth}
		\raisebox{-0.9\height}{\includegraphics[width=\textwidth]{Hoofdstukken/Optisch/pdf/klein_motorschip_anker.pdf}}
		\RemoveLine
		\caption{}
		\label{pic:optisch:klein_anker}
	\end{minipage}
	\hfill
	\begin{minipage}[t]{0.22\textwidth}
		\raisebox{-0.9\height}{\includegraphics[width=\textwidth]{Hoofdstukken/Optisch/pdf/klein_motorschip_anker_dag.pdf}}
		\RemoveLine
		\caption{}
		\label{pic:optisch:klein_anker2}
	\end{minipage}
\end{figure}%
\vspace{-0.6cm}%
Overdag draag een klein schip, evenals een groot schip, een zwarte bol op het voorschip op een goed zichtbare plaats. Dit is te zien in figuur \ref{pic:optisch:klein_anker2}.


\subsection{Overige tekens}
\label{par:optisch:overig}
% --- Gele ruit
\begin{figure}[H]
	\centering
	\begin{minipage}[t]{0.75\textwidth}
		\paragraph{Passagiers ruit}
		Wanneer een passagiersschip (meer dan 12 personen) een lengte bedraagt van minder dan 20 meter dient deze een gele ruit te dragen op een positie waar deze goed zichtbaar is. De ruit, te zien in figuur \ref{pic:optisch:gele_ruit}, maakt de voorrangspositie van het passagiersschip duidelijk naar de omgeving. 
	\end{minipage}
	\hfill
	\begin{minipage}[t]{0.22\textwidth}
		\raisebox{-0.9\height}{\includegraphics[width=\textwidth]{Hoofdstukken/Optisch/pdf/klein_motorschip_ruit.pdf}}
		\RemoveLine
		\caption{}
		\label{pic:optisch:gele_ruit}
	\end{minipage}
\end{figure}

% --- Zeilschip kegel
\begin{figure}[H]
	\centering
	\begin{minipage}[t]{0.75\textwidth}
		\paragraph{Zeilkegel}
		Wanneer een zeilschip (groot en klein) overdag op zowel een zeil vaart en gebruik maakt van een motor, is het vereist om een zwarte kegel te dragen. Deze kegel dient op een hoge plek geplaatst worden waar deze goed zichtbaar is. Deze kegel is te zien in figuur \ref{pic:optisch:zeilkegel}.
	\end{minipage}
	\hfill
	\begin{minipage}[t]{0.22\textwidth}
		\raisebox{-0.9\height}{\includegraphics[width=\textwidth]{Hoofdstukken/Optisch/pdf/klein_zeilschip_motor_dag.pdf}}
		\RemoveLine
		\caption{}
		\label{pic:optisch:zeilkegel}
	\end{minipage}
\end{figure}

% --- Schepen in bedrijf
\begin{figure}[H]
	\centering
	\begin{minipage}[t]{0.50\textwidth}
		\paragraph{In bedrijf zijnde werktuigen}
		Schepen die werk uitvoeren op het water (metingen, onderhoud, etc.) kunnen  zowel overdag als 's nachts aangeven welke zijdes langs het schip vrij zijn om te passeren. 
	\end{minipage}
	\hfill
		\begin{minipage}[t]{0.22\textwidth}
		\raisebox{-0.9\height}{\includegraphics[width=\textwidth]{Hoofdstukken/Optisch/pdf/werktuig.pdf}}
		\RemoveLine
		\caption{}
		\label{pic:optisch:in_bedrijf}
	\end{minipage}
	\hfill
	\begin{minipage}[t]{0.22\textwidth}
		\raisebox{-0.9\height}{\includegraphics[width=\textwidth]{Hoofdstukken/Optisch/pdf/werktuig_dag.pdf}}
		\RemoveLine
		\caption{}
		\label{pic:optisch:in_bedrijf2}
	\end{minipage}
\end{figure}
\vspace{-0.6cm}%
's Nachts geeft een rood rondom schijnend licht de zijde aan die niet vrij bevaarbaar is. De zijde waar wel langs gevaren mag worden wordt gemarkeerd met twee groene rondom schijnende lichten boven elkaar. In figuur \ref{pic:optisch:in_bedrijf} zijn deze lichten te zien in een vooraanzicht.

Overdag wordt gebruik gemaakt van bollen en ruiten voor het aangeven van vrije en niet-vrije zijdes. De niet vrije zijde is gemarkeerd met een rode bol en de vrije zijde wordt aangegeven met twee groene ruiten boven elkaar. In figuur \ref{pic:optisch:in_bedrijf2} is deze opstelling te zien.

% --- hinder
\begin{figure}[H]
	\centering
	\begin{minipage}[t]{0.75\textwidth}
		\paragraph{Vermijden hinderlijke waterbewegingen}
		Een schip wat werkzaamheden op het water uitvoert kan een teken dragen waarmee verplicht wordt voor omvarenden om hinderlijke waterbewegingen te vermijden. Dit wordt overdag met een rood-wit bord aangegeven en 's nachts met een rood en wit rondom schijnend licht onder elkaar. Zowel de dag als nacht situatie is te zien in figuur  \ref{pic:optisch:hinder}. De borden of lichten gelden alleen aan de zijde van het schip waar deze geplaatst zijn.
	\end{minipage}
	\hfill
	\begin{minipage}[t]{0.22\textwidth}
		\raisebox{-0.9\height}{\includegraphics[width=\textwidth]{Hoofdstukken/Optisch/pdf/werktuig_verbod_vaarbeweging.pdf}}
		\RemoveLine
		\caption{}
		\label{pic:optisch:hinder}
	\end{minipage}
\end{figure}

% --- Duiken
\begin{figure}[H]
	\centering
	\begin{minipage}[t]{0.75\textwidth}
		\paragraph{Duiktekens}
		Wanneer er vanaf een schip duiksport wordt uitgeoefend dient dit aangegeven te worden met de internationale seinvlag `A'. Deze vlag is te zien in figuur  \ref{pic:optisch:duik}. In het figuur is deze te zien in combinatie met een ankerbol, omdat dit een veel voorkomende samenstelling is. In de nacht dient de vlag duidelijk verlicht te worden.
	\end{minipage}
	\hfill
	\begin{minipage}[t]{0.22\textwidth}
		\raisebox{-0.9\height}{\includegraphics[width=\textwidth]{Hoofdstukken/Optisch/pdf/klein_motorschip_duiker.pdf}}
		\RemoveLine
		\caption{}
		\label{pic:optisch:duik}
	\end{minipage}
\end{figure}

\section{Conclusie}
Na het lezen van dit hoofdstuk heb je kennis van de optische tekens van schepen. Met deze kennis kun je schepen s' nachts aan hun navigatieverlichting herkennen en hierdoor de juiste voorrangsregels toepassen. Ook ken je enkele dagtekens en hun betekenis.
\header{6}
\chapter{Schiemannen}
\section{Inleiding}
Het leggen van knopen en werken met touwen wordt ook wel schiemannen genoemd. In dit hoofdstuk leer je de knopen die belangrijk zijn tijdens het varen. Daarnaast moet je begrijpen waarom en wanneer je een knoop gebruikt.
\section{De knopen}
\subsection{Halve steek \hfill \hspace{2 cm} \textit{Figuur \ref{pic:halve_steek}} } 
Een halve steek leg je wanneer je een touw vast wil leggen waar weinig kracht op komt. De halve steek is de basis voor veel knopen en steken.
\subsection{Slipsteek \hfill \textit{Figuur \ref{pic:slip_steek}}}
De slipsteek kan alleen gebruikt worden in situaties waar weinig kracht op de lijn komt. Het voordeel van een slipsteek is dat hij snel los te maken is.
\subsection{Achtknoop \hfill \textit{Figuur \ref{pic:achtknoop}}}
Een achtknoop wordt gebruikt om een verdikking in een touw te maken. Hiermee voorkom je bijvoorbeeld dat een touw door een blok schiet. 
\begin{figure}[h]
  \centering
  \begin{minipage}[b]{0.32\textwidth}
  \centering
    \includegraphics[width=0.8\textwidth]{Hoofdstukken/Schiemannen/pdf/halve_steek.pdf}
    \caption{Halve Steek}
    \label{pic:halve_steek}
  \end{minipage}
  \hfill
  \begin{minipage}[b]{0.32\textwidth}
    \centering
    \includegraphics[width=0.8\textwidth]{Hoofdstukken/Schiemannen/pdf/slip_steek.pdf}
    \caption{Slipsteek}
    \label{pic:slip_steek}
    \end{minipage}
  \hfill
  \begin{minipage}[b]{0.32\textwidth}
    \centering
    \includegraphics[width=0.8\textwidth]{Hoofdstukken/Schiemannen/pdf/achtknoop.pdf}
    \caption{Achtknoop}
    \label{pic:achtknoop}
  \end{minipage}
\end{figure}
\subsection{Platte knoop \hfill \textit{Figuur \ref{pic:platte_knoop}}}
Deze knoop is geschikt voor het verbinden van twee uiteinde van een touw van gelijke dikte. Deze knoop is niet geschikt voor situaties waar veel kracht op de lijn komt te staan. Hiervoor is een schootsteek beter geschikt.
\subsection{Schootsteek \hfill \textit{Figuur \ref{pic:schoot_steek}}}
Een schootsteek is geschikt om twee touwen van ongelijke dikte aan elkaar te maken. De knoop is ook geschikt voor touwen van gelijke dikte en kan veel kracht aan. Leg met het dikkere touw altijd de lus. Dat maakt de knoop makkelijker. 
\subsection{Mastworp \hfill \textit{Figuur \ref{pic:mastworp}}}
Deze knoop wordt veel gebruikt met pionieren en om je boot aan te leggen. De knoop trekt zichzelf strakker wanneer er meer kracht op komt.
\begin{figure}[h]
  \centering
  \begin{minipage}[b]{0.32\textwidth}
  \centering
    \includegraphics[width=0.8\textwidth]{Hoofdstukken/Schiemannen/pdf/platteknoop.pdf}
    \caption{Platte Knoop}
    \label{pic:platte_knoop}
  \end{minipage}
  \hfill
  \begin{minipage}[b]{0.32\textwidth}
    \centering
    \includegraphics[width=0.8\textwidth]{Hoofdstukken/Schiemannen/pdf/schootsteek.pdf}
    \caption{Schootsteek}
    \label{pic:schoot_steek}
    \end{minipage}
  \hfill
  \begin{minipage}[b]{0.32\textwidth}
    \centering
    \includegraphics[width=0.8\textwidth]{Hoofdstukken/Schiemannen/pdf/mastworp.pdf}
    \caption{Mastworp}
    \label{pic:mastworp}
  \end{minipage}
\end{figure}
\subsection{Paalsteek \hfill \textit{Figuur \ref{pic:paal_steek}}} 
De paalsteek is bedoeld om een lus in een touw te leggen. De lus is erg sterk, maar kan wel gemakkelijk weer losgehaald worden.
\subsection{Dubbele halve steek \hfill \textit{Figuur \ref{pic:dub_halve_steek}}}
Een dubbele halve steek is geschikt om lijnen strak aan een oog vast te maken. Dit is bijvoorbeeld handig als je aan wilt leggen met een meerpen. Je wikkelt eerst de lijn tweemaal om een oog en legt er vervolgens twee halve steken in. Dit maakt een mastworp. Je kan de eerste halve steek ook vervangen door een slipsteek.
\subsection{Een tros opschieten \hfill \textit{Figuur \ref{pic:opschieten}}}
Een tros opschieten is een manier om een lijn op te bergen zonder dat deze in de knoop raakt. Tijdens het opschieten maak je een aantal gelijke lussen. Aan het einde wikkel je de rest van het touw om de lussen en leg je een knoop als in het figuur. Opschieten staat ook wel bekend als opbossen.
\begin{figure}[h]
  \centering
  \begin{minipage}[b]{0.32\textwidth}
  \centering
    \includegraphics[width=0.8\textwidth]{Hoofdstukken/Schiemannen/pdf/paalsteek.pdf}
    \caption{Paalsteek}
    \label{pic:paal_steek}
  \end{minipage}
  \hfill
  \begin{minipage}[b]{0.32\textwidth}
    \centering
    \includegraphics[width=0.8\textwidth]{Hoofdstukken/Schiemannen/pdf/dubble_halve_steek.pdf}
    \caption{Dubbele halve steek}
    \label{pic:dub_halve_steek}
    \end{minipage}
  \hfill
   \begin{minipage}[b]{0.32\textwidth}
    \centering
    \includegraphics[width=0.8\textwidth]{Hoofdstukken/Schiemannen/pdf/opbossen.pdf}
    \caption{Opschieten}
    \label{pic:opschieten}
    \end{minipage}
\end{figure}
\newpage
\subsection{Een kikker beleggen \hfill \textit{Figuur \ref{pic:kikker1}, \ref{pic:kikker2} \& \ref{pic:kikker3}}}
Wanneer je een kikker belegt, leg je een lijn vast op een kikker. Dit is nodig voor bijvoorbeeld het hijsen van het zeil. Belangrijk bij het beleggen van een kikker in een boot is dat je de ''eindlus'' aan de bovenzijde van de kikker legt. Anders kan deze er af vallen en de kikker los raken. Daarnaast moet je het lusje zo draaien dat het uiteinde weer in de richting van het vorige achtje gaat.
\begin{figure}[h]
  \centering
  \begin{minipage}[b]{0.32\textwidth}
  \centering
    \includegraphics[width=0.8\textwidth]{Hoofdstukken/Schiemannen/pdf/kikker1.pdf}
    \caption{Kikker 8'tjes}
    \label{pic:kikker1}
  \end{minipage}
  \hfill
  \begin{minipage}[b]{0.32\textwidth}
    \centering
    \includegraphics[width=0.8\textwidth]{Hoofdstukken/Schiemannen/pdf/kikker2.pdf}
    \caption{Kikker eind lus}
    \label{pic:kikker2}
    \end{minipage}
  \hfill
   \begin{minipage}[b]{0.32\textwidth}
    \centering
    \includegraphics[width=0.8\textwidth]{Hoofdstukken/Schiemannen/pdf/kikker3.pdf}
    \caption{Kikker afknopen}
    \label{pic:kikker3}
    \end{minipage}
\end{figure}
\section{Conclusie}
Na het lezen van dit hoofdstuk en het oefenen met de knopen, snap je het nut en toepassing van de verschillende knopen. Ook kan je alle knopen zonder voorbeeld leggen. Een instructeur kan dit in de onderstaande tabel aftekenen.
\vspace{2cm}
\begin{table}[H]
\centering
\caption{Aftekenen knopen}
\label{my-label}
\begin{tabular}{|l|l|l|}
\hline
\textbf{Knoop of Handeling}  & \textbf{Paraaf} & \textbf{Paraaf} \\ \hline
\textit{Halve Steek}         &                 &                 \\ \hline
\textit{Slipsteek}          &                 &                 \\ \hline
\textit{Achtknoop}           &                 &                 \\ \hline
\textit{Platte Knoop}        &                 &                 \\ \hline
\textit{Schootsteek}        &                 &                 \\ \hline
\textit{Mastworp}            &                 &                 \\ \hline
\textit{Paalsteek}           &                 &                 \\ \hline
\textit{Dubbele halve steek} &                 &                 \\ \hline
\textit{Tros opschieten}     &                 &                 \\ \hline
\textit{Kikker beleggen}     &                 &                 \\ \hline
\end{tabular}
\end{table}
\header{8}
\chapter{Krachten op het schip}
\section{Inleiding}

Om optimaal gebruik van de wind te kunnen maken, is het van belang om de krachten en hun effecten op het schip te begrijpen. Door deze kennis juist toe te passen is het mogelijk om sneller en efficiënter te zeilen. De juiste kennis van krachten is ook handig voor een aantal zeilmanoeuvres. 

\section{Krachten en koppels}
\label{par:krachten_uitleg}
In dit hoofdstuk wordt gebruik gemaakt van twee eenvoudige natuurkundige principes: krachten en koppels. De theorie hierachter wordt daarom kort toegelicht.

\begin{figure}[H]
	\centering
	\begin{minipage}[t]{0.75\textwidth}
		\paragraph{Kracht}
		Wanneer een kracht op een voorwerp gezet wordt kunnen er twee dingen gebeuren: het voorwerp verplaatst of het voorwerp vervormt. In dit hoofdstuk wordt alleen gekeken naar het verplaatsende effect van een kracht. Een kracht is gedefinieerd door twee componenten: een richting en een grootte/sterkte. Een voorbeeld van een kracht op een boot is te zien in figuur  \ref{pic:kracht}. 
	\end{minipage}
	\hfill
	\begin{minipage}[t]{0.22\textwidth}
		\raisebox{-0.9\height}{\includegraphics[width=\textwidth,]{Hoofdstukken/Krachten/pdf/kracht_op_vlet.pdf}}
		\RemoveLine
		\caption{}
		\label{pic:kracht}
	\end{minipage}
\end{figure} 

\paragraph{Krachten optellen en ontbinden}
Met krachten kan ook `gerekend' worden. Zo kunnen twee krachten die op hetzelfde voorwerp werken worden opgeteld. Een enkele kracht kan daarentegen worden opgedeeld in meerdere krachten met hetzelfde effect.

In figuur \ref{pic:kracht_som} zijn twee krachten te zien die op hetzelfde punt werken: blauw en rood. Door een parallellogram te maken van de krachten, is het mogelijk om de zwarte kracht te krijgen. Deze kracht heeft hetzelfde effect als de rode en blauwe kracht bij elkaar.

Het tegenovergestelde is ook mogelijk. Wanneer je een enkele kracht hebt, kun je deze splitsen in twee krachten met hetzelfde effect. In figuur \ref{pic:kracht_splits} is een enkele zwarte krachtpijl te zien. Deze wordt opgesplitst in de richting van rood en blauw.

Door een parallellogram om de zwarte kracht heen te tekenen, is het mogelijk lengte van de rode en blauwe kracht te bepalen. De zijdes vanuit de punt geven de krachten weer die hetzelfde effect hebben als de zwarte kracht.


\begin{figure}[H]
	\centering
	\begin{minipage}[b]{0.32\textwidth}
		\centering
		\includegraphics[width=0.85\textwidth]{Hoofdstukken/Krachten/pdf/kracht_optellen.pdf}
		\caption{Optellen}
		\label{pic:kracht_som}
	\end{minipage}
	\hfill
	\begin{minipage}[b]{0.32\textwidth}
		\centering
		\includegraphics[width=0.85\textwidth]{Hoofdstukken/Krachten/pdf/kracht_richtingen.pdf}
		\caption{Enkele kracht}
		\label{pic:kracht_splits}
	\end{minipage}
	\hfill
	\begin{minipage}[b]{0.32\textwidth}
		\centering
		\includegraphics[width=0.85\textwidth]{Hoofdstukken/Krachten/pdf/kracht_splitsen.pdf}
		\caption{Gesplitste kracht}
		\label{pic:kracht_splits2}
	\end{minipage}
\end{figure}


\paragraph{Koppels en armen}
Een koppel is de natuurkundige beschrijving voor de kracht die een voorwerp doet draaien. In het voorbeeld in figuur \ref{pic:koppel} is een balk met een draaipunt te zien. Door een kracht op het einde van de balk te zetten zal er een draaiing ontstaan. Het effect van de kracht is een koppel om het draaipunt.

\begin{figure}[H]
	\centering
	\begin{minipage}[b]{0.49\textwidth}
		\centering
		\includegraphics[width=0.75\textwidth]{Hoofdstukken/Krachten/pdf/kracht_koppel.pdf}
		\caption{Koppel}
		\label{pic:koppel}
	\end{minipage}
	\hfill
	\begin{minipage}[b]{0.49\textwidth}
		\centering
		\includegraphics[width=0.75\textwidth]{Hoofdstukken/Krachten/pdf/kracht_arm.pdf}
		\caption{Arm}
		\label{pic:arm}
	\end{minipage}
\end{figure}
De afstand tussen de kracht en het draaipunt wordt ook wel de arm genoemd. Door de arm te vergroten is het mogelijk om met dezelfde kracht, een sterkere koppel te maken. Dit is weergegeven in figuur \ref{pic:arm}. De arm versterkt hier de kracht die op de balk geplaatst wordt.

\section{Voortstuwing van de boot}
Nu de basiskennis over de krachten is verworven, kan er verder gekeken worden naar hoe een boot voortgestuwd wordt. Dit wordt in drie stappen toegelicht: voortstuwende werking van de zeilen, driftbeperkende middelen en tot slot voortstuwing.

\subsection*{Voortstuwende werking van de zeilen}
De voortstuwende werking van de zeilen heeft alles te maken met de stroming van de wind langs de zeilen. In figuur \ref{pic:stroming} is een versimpeling van deze stroming te zien langs de fok en het grootzeil. 

Door de bolle vorm van de zeilen moet de wind aan de bolle zijde een langere weg afleggen dan aan de holle zijde. Hierdoor zal de wind aan de bolle zijde sneller stromen dan de holle zijde. Sneller stromende lucht heeft een lagedruk. Hierdoor ontstaat een drukverschil zoals te zien is in figuur \ref{pic:druk}.

\begin{figure}[H]
	\centering
	\begin{minipage}[b]{0.32\textwidth}
		\centering
		\includegraphics[width=0.95\textwidth]{Hoofdstukken/Krachten/pdf/voortstuwing_zeil_stroming.pdf}
		\caption{Stroming}
		\label{pic:stroming}
	\end{minipage}
	\hfill
	\begin{minipage}[b]{0.32\textwidth}
		\centering
		\includegraphics[width=0.95\textwidth]{Hoofdstukken/Krachten/pdf/voortstuwing_zeil_druk.pdf}
		\caption{Druk}
		\label{pic:druk}
	\end{minipage}
	\hfill
	\begin{minipage}[b]{0.32\textwidth}
		\centering
		\includegraphics[width=0.95\textwidth]{Hoofdstukken/Krachten/pdf/voortstuwing_zeil_krachten_los.pdf}
		\caption{Voortstuwing}
		\label{pic:stuwing}
	\end{minipage}
\end{figure}

Omdat de lucht van het hoge naar het lage druk gebied toe wil, ontstaat er een kracht op beide zeilen. Wanneer je per zeil alle krachten opsomt, krijg je de krachten te zien in figuur \ref{pic:stuwing}.

\begin{figure}[H]
	\centering
	\begin{minipage}[t]{0.63\textwidth}
		\vspace*{0.2cm}
		Om makkelijk met deze krachten te werken nemen we de kracht van zowel de fok als het grootzeil samen. Dit levert de kracht in figuur \ref{pic:zeilpunt} op en heet de zeilkracht. Het punt waaruit de zeilkracht werkt noemen we het zeilpunt. 
	\end{minipage}
	\hfill
	\begin{minipage}[t]{0.32\textwidth}
		\raisebox{-0.9\height}{\includegraphics[width=0.95\textwidth]{Hoofdstukken/Krachten/pdf/voortstuwing_zeil_zeilkracht.pdf}}
		\RemoveLine
		\caption{Zeilpunt}
		\label{pic:zeilpunt}
	\end{minipage}
\end{figure} 

\subsection{Driftbeperkende middelen}
Om te voorkomen dat de zeilkracht de boot enkel verlijerd of drift, heeft deze driftbeperkende middelen. Deze middelen maken het lastiger voor de boot om zijwaarts over het water te bewegen doordat deze dwars op de boot staan. Een lelievlet beschikt over vier driftbeperkende middelen: het onderwaterschip, het zwaard, de scheg en het roer. In figuur \ref{pic:drift_beperking} zijn deze onderdelen afgebeeld.

\begin{figure}[H]
	\centering
	\begin{minipage}[b]{0.48\textwidth}
		\centering
		\includegraphics[width=0.90\textwidth]{Hoofdstukken/Krachten/pdf/lateraal_driftbeperking.pdf}
		\caption{Driftbeperking}
		\label{pic:drift_beperking}
	\end{minipage}
	\hfill
	\begin{minipage}[b]{0.48\textwidth}
		\centering
		\includegraphics[width=0.90\textwidth]{Hoofdstukken/Krachten/pdf/lateraal_boven.pdf}
		\caption{Zijwaartse kracht}
		\label{pic:zijwaarts}
	\end{minipage}
\end{figure}
Om de driftbeperking en de krachten die hierbij komen kijken beter te begrijpen, kijken we naar de volgende situatie: een boot wordt door een externe kracht zijwaarts over het water geduwd. Dit is uitgebeeld in figuur \ref{pic:zijwaarts}. De zwarte pijl stelt het duwen van de boot voor. 

Wanneer de boot zijwaarts verplaatst wordt, zullen de driftbeperkende middelen dit tegengaan omdat deze het water om zich heen moeten verplaatsen. In figuur \ref{pic:zijwaarts} is de weerstand die deze middelen geven met de rode pijlen uitgebeeld. Figuur \ref{pic:zijwaarts_totaal} toont een zijaanzicht van deze situatie.

\begin{figure}[H]
	\centering
	\begin{minipage}[b]{0.48\textwidth}
		\centering
		\includegraphics[width=0.90\textwidth]{Hoofdstukken/Krachten/pdf/lateraal_zij.pdf}
		\caption{Zijwaartse kracht zijaanzicht}
		\label{pic:zijwaarts_totaal}
	\end{minipage}
	\hfill
	\begin{minipage}[b]{0.48\textwidth}
		\centering
		\includegraphics[width=0.90\textwidth]{Hoofdstukken/Krachten/pdf/lateraal_punt.pdf}
		\caption{Lateraalpunt}
		\label{pic:lateraal}
	\end{minipage}
\end{figure}

Wanneer we al deze krachten sommeren krijgen we een totaal kracht (rood) die is afgebeeld in figuur \ref{pic:lateraal}. Deze kracht werkt vanuit een punt dat ook wel het lateraalpunt genoemd wordt. Het lateraalpunt is formeel gedefinieerd als: ``\textit{Het punt waar alle laterale (zijwaartse) krachten van het schip op werken}''. Het lateraalpunt is tevens het draaipunt van het schip.

Door middel van het lateraalpunt kan makkelijk gerekend worden met de driftbeperkende kracht van het schip. Dit zal van pas komen in het begrijpen van de voortstuwing van de boot.

\newpage
\subsection{Voortstuwing}
Met de kennis die is opgedaan over zowel de zeil- als driftbeperkende kracht is het mogelijk om te kijken naar hoe de boot vooruit bewogen wordt. In figuur \ref{pic:krachten_los} is een boot te zien samen met een zeilkracht (blauw) en een driftbeperkende kracht (rood). Deze krachten zijn getekend vanuit het zeilpunt en het lateraalpunt.

\begin{figure}[H]
	\centering
	\begin{minipage}[b]{0.48\textwidth}
		\centering
		\includegraphics[width=0.90\textwidth]{Hoofdstukken/Krachten/pdf/voortstuwing_krachten.pdf}
		\caption{Zeil en drifbeperkende kracht}
		\label{pic:krachten_los}
	\end{minipage}
	\hfill
	\begin{minipage}[b]{0.48\textwidth}
		\centering
		\includegraphics[width=0.90\textwidth]{Hoofdstukken/Krachten/pdf/voortstuwing_totaal.pdf}
		\caption{Voortstuwende kracht}
		\label{pic:voorwaardse_kracht}
	\end{minipage}
\end{figure}

In figuur \ref{pic:voorwaardse_kracht} wordt de zeilkracht gesplitst in twee krachten met hetzelfde effect: de voorwaartse kracht (groen) en de verlijerende kracht (rood). De verlijerende kracht wordt echter opgeheven door de driftbeperkende kracht. Hierdoor blijft alleen de voortstuwende kracht nog over. Deze samenwerking van de krachten geeft een zeilboot zijn voortstuwing.

\section{Correcte zeilstand}
Door de zeilen in een correcte stand te zetten is het mogelijk de voortstuwende werking hiervan te optimaliseren. Dit heet ook wel het trimmen van de zeilen. Een correct getrimd zeil is weergegeven in figuur \ref{pic:zeil_goed}. In dit figuur is te zien dat de stroming van de wind de zeilen strak volgt. Deze stroming creëert zo het grootste drukverschil en hiermee de meeste voortstuwing. 

Wanneer het zeil echter te strak staat, ontstaat de situatie in figuur \ref{pic:zeil_strak}. De windstroming laat halverwege het grootzeil `los'. Dit zorgt voor turbulentie aan het achterlijk van het zeil en deze turbulentie verlaagt het drukverschil tussen beide zijde van het zeil. Dit zorgt vervolgens voor een lagere voortstuwing.

\begin{figure}[H]
  \centering
  \begin{minipage}[b]{0.32\textwidth}
  \centering
    \includegraphics[height=5cm]{Hoofdstukken/Krachten/pdf/trimmen_goed.pdf}
    \caption{Zeil goed}
    \label{pic:zeil_goed}
  \end{minipage}
  \hfill
  \begin{minipage}[b]{0.32\textwidth}
    \centering
    \includegraphics[height=5cm]{Hoofdstukken/Krachten/pdf/trimmen_strak.pdf}
    \caption{Zeil te strak}
    \label{pic:zeil_strak}
    \end{minipage}
  \hfill
  \begin{minipage}[b]{0.32\textwidth}
    \centering
    \includegraphics[height=5cm]{Hoofdstukken/Krachten/pdf/trimmen_los.pdf}
    \caption{Zeil te los}
    \label{pic:zeil_los}
  \end{minipage}
\end{figure}

In figuur \ref{pic:zeil_los} is een te los zeil te zien. De windstromen aan de hoge kant van het zeil volgen het zeil niet strak en aan de lage zijde wordt de stroming omgeleid door het zeil. Deze verstoringen in de stroming zorgen voor een lager drukverschil en minder voortstuwing. Dit wordt ook wel het killen van het zeil genoemd.

Een goede manier om je zeil te trimmen is om hem net zo lang te laten vieren totdat er een kleine tegenbolling in het voorlijk te zien valt. Daarna trek je het zeil weer een klein beetje aan tot deze weg valt. Op deze manier benut je de wind maximaal. Door dit regelmatig te doen weet je zeker dat je optimaal zeilt. 

\section{Effecten van de fok en het grootzeil}
Het grootzeil en de fok hebben allebei een ander effect op de koers van boot. Het verschil in dit gedrag is te verklaren door hun positie ten opzichte van het lateraalpunt. Omdat het grootzeil (en het bijbehorende zeilpunt) zich achter het lateraalpunt bevindt, creëert het grootzeil een oploevend koppel. Dit effect is weergegeven in figuur \ref{pic:effect_grootzeil}.

Het omgekeerde is waar voor de fok. Doordat deze zich voor het lateraalpunt bevindt, ontstaat er een afvallend koppel. Figuur \ref{pic:effect_fok} toont deze situatie. 

\begin{figure}[ht]
	\centering
	\begin{minipage}[b]{0.49\textwidth}
		\centering
		\includegraphics[width=\textwidth]{Hoofdstukken/Krachten/pdf/effect_grootzeil.pdf}
		\caption{Effect van het grootzeil}
		\label{pic:effect_grootzeil}
	\end{minipage}
	\hfill
	\begin{minipage}[b]{0.49\textwidth}
		\centering
		\includegraphics[width=\textwidth]{Hoofdstukken/Krachten/pdf/effect_fok.pdf}
		\caption{Effect van de fok}
		\label{pic:effect_fok}
	\end{minipage}
\end{figure}

Door gebruikt te maken van de sturende werking van de zeilen is het mogelijk de boot van koers te veranderen met minder gebruik van het roer. Hiermee wordt het oploeven en afvallen efficiënter.

\section{Effect van de helling}
Een lelievlet is van nature loefgierig. Dit houdt in dat de boot met een normale zeilstand maar zonder roer, de wind in draait. Door de helling van de boot te veranderen is het mogelijk om de loef- en lijgierigheid hiervan aan te passen.

De loefgierigheid van de boot komt voort uit de positie van het zeilpunt ten opzichte van het lateraalpunt. Dit is duidelijk te zien in het achteraanzicht van een boot in figuur \ref{pic:helling_vlak_achter}. Tussen het zeilpunt en het lateraalpunt bevindt zich een arm. Omdat het lateraalpunt tevens het draaipunt van de boot is, zorgt deze arm voor een koppel om het lateraalpunt waardoor de boot wil oploeven. Dit koppel is weergegeven in figuur \ref{pic:helling_valk_boven}. De arm tussen het zeilpunt en het lateraalpunt geeft de boot dus zijn loefgierheid.

\begin{figure}[H]
	\centering
	\begin{minipage}[b]{0.48\textwidth}
		\centering
		\includegraphics[height=5cm]{Hoofdstukken/Krachten/pdf/helling_vlak_achter.pdf}
		\caption{Zeilpunt en lateraalpunt}
		\label{pic:helling_vlak_achter}
	\end{minipage}
	\hfill
	\begin{minipage}[b]{0.48\textwidth}
		\centering
		\includegraphics[height=5cm]{Hoofdstukken/Krachten/pdf/helling_vlak_boven.pdf}
		\caption{Oploevend koppel}
		\label{pic:helling_valk_boven}
	\end{minipage}
\end{figure}


De helling van de boot heeft een invloed op de lengte van deze arm. Door de boot meer naar de lijkant te hellen wordt de arm vergroot. Dit zorgt vervolgens voor een groter oploevend koppel. Figuur \ref{pic:helling_oploeven} laat deze vergrootte arm zien. 

\begin{figure}[H]
	\centering
	\begin{minipage}[b]{0.48\textwidth}
		\centering
		\includegraphics[height=5cm]{Hoofdstukken/Krachten/pdf/helling_lij_achter.pdf}
		\includegraphics[height=5cm]{Hoofdstukken/Krachten/pdf/helling_lij_boven.pdf}
		\caption{Helling naar lij}
		\label{pic:helling_oploeven}
	\end{minipage}
	\hfill
	\begin{minipage}[b]{0.48\textwidth}
		\centering
		\includegraphics[height=5cm]{Hoofdstukken/Krachten/pdf/helling_loef_achter.pdf}
		\includegraphics[height=5cm]{Hoofdstukken/Krachten/pdf/helling_loef_boven.pdf}
		\caption{Helling naar loef}
		\label{pic:helling_afvallen}
	\end{minipage}
\end{figure}

Het tegenovergestelde is ook mogelijk: door de boot te hellen naar de loefzijde, wordt de arm verkleind of zelfs omgedraaid. Dit zorgt ervoor dat de boot lijgierig wordt. Dit wordt weergegeven in figuur \ref{pic:helling_afvallen}.


Door correct gebruik te maken van de helling van het schip, is het mogelijk sneller en efficiënter af te vallen en op te loeven. Tijdens complexe manoeuvres biedt dit extra controle over de boot.  

\section{Stabiliteit}
De zeilkracht die de boot voortstuwt heeft ook een nadelige bijwerking: de zeilkracht wil de boot ook om duwen. De zeilkracht creëert een koppel rondom het lateraalpunt genaamd het kenterend koppel. Het koppel is weergegeven in figuur \ref{pic:kenterende_koppel}. Een boot slaat echter niet zomaar om omdat deze stabiel is. Deze stabiliteit komt voort uit een evenwicht tussen kenterend koppel en zijn tegenhanger het oprichtend koppel. 

Het oprichtend koppel ontstaat vanuit twee krachten die werken vanuit twee punten:
\begin{enumerate}
	\item \textbf{Drukpunt:} Dit is het aangrijppunt van alle opwaartse kracht op de boot. Ook wel het drijfpunt genoemd.
	\item \textbf{Zwaartepunt:} Dit is het aangrijppunt voor alle zwaartekracht die op de boot werkt.
\end{enumerate}

In figuur \ref{pic:oprichtend} zijn de punten met hun krachten geïllustreerd. Het effect hiervan is het oprichtende koppel. Wanneer de boot verder helt, zoals in figuur \ref{pic:oprichtend_helling} is te zien dat de arm tussen het zwaartepunt en het drukpunt groter wordt. Hierdoor wordt ook het oprichtend koppel verstrekt. 

De boot in figuur \ref{pic:oprichtend_helling} is een gewichtsstabiele boot. Deze categorie boten hebben een laag zwaartepunt door hun bouw of een kiel. Deze boten hebben bij een kleine helling een lage stabiliteit. Naar mate de helling toeneemt, neemt de stabiliteit ook toe. 

\begin{figure}[H]
	\centering
	\begin{minipage}[b]{0.32\textwidth}
		\centering
		\includegraphics[height=6cm]{Hoofdstukken/Krachten/pdf/stabiliteit_kenternd.pdf}
		\caption{Kenterend koppel}
		\label{pic:kenterende_koppel}
	\end{minipage}
	\hfill
	\begin{minipage}[b]{0.33\textwidth}
		\centering
		\includegraphics[height=6cm]{Hoofdstukken/Krachten/pdf/stabiliteit_gewichtstabiel.pdf}
		\caption{Oprichtend koppel}
		\label{pic:oprichtend}
	\end{minipage}
	\hfill
	\begin{minipage}[b]{0.32\textwidth}
		\centering
		\includegraphics[height=6cm]{Hoofdstukken/Krachten/pdf/stabiliteit_gewichtstabiel_helling.pdf}
		\caption{Helling}
		\label{pic:oprichtend_helling}
	\end{minipage}
\end{figure}

De boot in figuur \ref{pic:vormstabiel} is een vormstabiele boot. Deze wordt gekenmerkt door een hoog zwaartepunt. Dit type schepen heeft een hoge stabiliteit bij een kleine helling. Naarmate de helling toeneemt, zoals in figuur \ref{pic:vormstabiel_helling}, komt het drukpunt dichter naar het zwaartepunt toe. Dit verkleint de arm en hierdoor neemt het oprichtend koppel af en dus ook de stabiliteit.
\begin{figure}[H]
	\centering
	\begin{minipage}[b]{0.49\textwidth}
		\centering
		\includegraphics[height=6cm]{Hoofdstukken/Krachten/pdf/stabiliteit_vormstabiel.pdf}
		\caption{Vormstabiel}
		\label{pic:vormstabiel}
	\end{minipage}
	\hfill
	\begin{minipage}[b]{0.49\textwidth}
		\centering
		\includegraphics[height=6cm]{Hoofdstukken/Krachten/pdf/stabiliteit_vormstabiel_helling.pdf}
		\caption{Vormstabiel onder helling}
		\label{pic:vormstabiel_helling}
	\end{minipage}
\end{figure}

Stabiliteit van schepen wordt dus gedicteerd door de ligging van het zwaartepunt ten opzichte van het drukpunt. Gewichtsstabiele schepen hebben daarom en lage beginstabiliteit en een hoge eindstabiliteit. Voor vormstabiele schepen is dit andersom. Deze hebben een hoge beginstabiliteit en een lage eindstabiliteit.  

\newpage
\section{Schijnbare wind}
De wind die op een boot werkt bestaat in feiten uit twee delen: de ware wind en de vaartwind.

\begin{enumerate}
	\item De \textbf{ware wind} ontstaat door drukverschillen in de atmosfeer en voel je als je stil staat. De ware wind is ook af te zien aan vlaggen. 
	\item De \textbf{vaartwind} ontstaat daarentegen door de voorwaartse snelheid van het schip. Dit fenomeen is ook te voelen als je op een windstille dag fiets: door de voortbeweging voel je toch wind. 
\end{enumerate}

De ware wind en de vaartwind vormen samen de schijnbare wind. Dit is de wind waar het schip op vooruit gaat en de wind die je in de boot voelt. In figuur \ref{pic:schijnbare_wind} is de schijnbare wind te zien en hoe deze ontstaat vanuit zijn twee componenten. 

\begin{figure}[H]
	\centering
	\begin{minipage}[b]{0.49\textwidth}
		\centering
		\includegraphics[height=6cm]{Hoofdstukken/Krachten/pdf/wind_schijnbaar.pdf}
		\caption{Schijnbare wind}
		\label{pic:schijnbare_wind}
	\end{minipage}
	\hfill
	\begin{minipage}[b]{0.49\textwidth}
		\centering
		\includegraphics[height=6cm]{Hoofdstukken/Krachten/pdf/wind_vlaag.pdf}
		\caption{Windvlaag}
		\label{pic:windvlaag}
	\end{minipage}
\end{figure}

De samenwerking van de ware wind en de vaartwind geeft een bijzondere dynamiek aan de schijnbare wind. Wanneer de ware wind plots krachtiger wordt, bijvoorbeeld door een windvlaag, zal de wind wat `ruimer' de boot in komen. Dit is te zien in figuur \ref{pic:windvlaag}. Doordat de wind wat ruimer is, is het mogelijk wat hoger te varen. Zo is het mogelijk om tijdens een windvlaag wat hoogte te winnen.

Een tweede effect is dat wanneer de snelheid van het schip toeneemt, en dus ook de vaartwind, de wind juist scherper de boot in komt. Snel varende schepen kunnen daarom wat minder hoog aan de wind varen.

\newpage
\section{Krachten op het roer}
Het roer maakt net als het zeil een kracht door drukverschillen. Deze kracht staat haaks op het roer en is te zien in figuur \ref{pic:roer}. De roerkracht wordt groter naar mate de snelheid van het schip toeneemt, of het roer verder uit staat.

\begin{figure}[H]
	\centering
	\begin{minipage}[b]{0.32\textwidth}
		\centering
		\includegraphics[height=2.75cm]{Hoofdstukken/Krachten/pdf/roer.pdf}
		\caption{Roerkracht}
		\label{pic:roer}
	\end{minipage}
	\hfill
	\begin{minipage}[b]{0.32\textwidth}
		\centering
		\includegraphics[height=2.75cm]{Hoofdstukken/Krachten/pdf/roer_kwart.pdf}
		\caption{Roer Half}
		\label{pic:roer_kwart}
	\end{minipage}
	\hfill
	\begin{minipage}[b]{0.32\textwidth}
		\centering
		\includegraphics[height=2.75cm]{Hoofdstukken/Krachten/pdf/roer_plat.pdf}
		\caption{Roer Plat}
		\label{pic:roer_plat}
	\end{minipage}
\end{figure}

Niet alle roerkracht wordt echter omgezet in een sturende beweging. Wanneer we net als bij het zeil de kracht van het roer ontbinden krijgen we twee krachten: de sturende kracht en de remmende kracht. Dit is te zien in figuur \ref{pic:roer_kwart}. Bij een hoek van 45 graden is de sturende kracht ongeveer even groot als de remmende kracht. 

Wanneer we het roer echter verder uitzetten en bijna plat doen, wordt de roerkracht groter. Maar als we de krachten ontbinden zoals in figuur \ref{pic:roer_plat}, is te zien dat de sturende kracht maar een klein beetje grote wordt dan in figuur \ref{pic:roer_kwart}. De remmende kracht neemt daarentegen wel veel toe en wordt groter dan de sturende kracht.

Bij het gebruik van het moet daarom altijd een afweging gemaakt worden tussen snelheid van het sturen en de snelheid die verloren wordt. 

\section{Conclusie}
In dit hoofdstuk zijn de verschillende krachten op het schip behandeld. Je snapt nu onder andere hoe een schip aan zijn voortstuwing komt, wat het effect van het grootzeil en de fok is en wat de invloed van de helling op de boot is. Met deze kennis het is mogelijk meer controle over de boot te krijgen.  

\makeatletter\@addtoreset{chapter}{part}
\makeatother 


\part{Oefenvragen}
\label{part:oefen}
\thispagestyle{empty}
\makeatletter\@openrightfalse
\makeatother
\setcounter{chapter}{0}
\usechapterimagefalse
\chapter{Bootonderdelen \& Zeiltermen}
\vspace{-120px}

\question{1}{Hoe heet de bovenste rand van het casco?}
\answerTextFour{Boeisel}{Berghout}{Dolboord}{Kim}
\question{2}{Hoe heet het opstaande randje op het voordek waar de voorstag aan vast zit?}
\answerTextFour{Sleepoog}{Hanekam}{Boeisel}{Voorstaghouder}
\question{3}{Tegen de wind in zeilen door steeds overstag te gaan heet:}
\answerTextFour{Kruisrak}{Opkruisen}{Laveren}{Alle bovenstaande antwoorden zijn goed}

\question{4}{Hoe heet de hoek van het zeil aangegeven in het plaatje?}
\vspace*{-1cm}
\answerTextPicture{Tophoek}{Halshoek}{Klauwhoek}{Schoothoek}{Hoofdstukken/Oefenvragen/pdf/tophoek.pdf}


\question{5}{Welk onderdeel wijst de pijl aan?}
\answerTextPicture{Roerkoningen}{Roerhaakgaten}{Roerpennen}{Vingerlingen}{Hoofdstukken/Oefenvragen/pdf/vingerlingen.pdf}

\question{6}{Welke stelling is waar?}
\answerTextPicture{I is de hogerwal en III is de lage kant}{II is de hogerwal en III is lagerwal}{IV is de lagerwal en II is de hogerwal}{II is de lage kant en III is de hoge kant}{Hoofdstukken/Oefenvragen/pdf/wallen.pdf}

\question{7}{Met welk lijntje hijs je de fok?}
\answerTextFour{Fokkeval}{Fokkelijn}{Fokkeschoot}{Grootschoot}
\newpage
\question{8}{Op welke afbeelding vaart de boot ruime wind?}
\answerPicture{Hoofdstukken/Oefenvragen/pdf/aan_de_wind.pdf}{Hoofdstukken/Oefenvragen/pdf/voor_de_wind.pdf}{Hoofdstukken/Oefenvragen/pdf/ruime_wind.pdf}{Hoofdstukken/Oefenvragen/pdf/halve_wind.pdf}


\question{9}{Welke stelling is waar?}%boven en benede winds
\answerTextPicture{De boot rondt de boei bovenwinds over bakboord}{De boot rondt de boei bovenwinds over stuurboord}{De boot rondt de boei benedenwinds over bakboord}{De boot rondt de boei benedenwinds over stuurboord.}{Hoofdstukken/Oefenvragen/pdf/bovenwinds_over_stuurboord_ronden.pdf}

\question{10}{Wat gebeurt er met je boot als je gaat oploeven?}%boven en benede winds
\answerTextFour{Je boot draait van de wind af}{Je draait door de wind heen}{Je draait naar de wind toe}{Je boot ligt in de wind}



\chapter{Veiligheid, Weer \& Vaarproblematiek}
\vspace{-120px}

\question{1}{Wat is geen eis aan een reddingsvest?}
\answerTextFour{Oranje of rood van kleur zijn}{Naam en adres van de fabrikant bevatten}{Handvatten hebben waarmee iemand uit het water getild kan worden}{Binnen 7 seconden op je rug draaien}

\question{2}{Wat is een gedragsregel op het water?}
\answerTextFour{Houd de schippersgroet in ere}{Kom niet op andermans schip zonder toestemming}{Houd je schip en omgeving schoon}{A, B en C zijn alledrie juist}

\question{3}{Wanneer de wind plots snel draait kan dit wijzen op:}
\answerTextFour{Een weersomslag}{Opkomende bewolking}{Groter wordende golven}{A, B en C zijn alledrie juist}

\question{4}{Waarom moet je bij je boot blijven als die is omgeslagen}
\answerTextFour{Omdat naar de kant zwemmen gevaarlijk is}{Omdat je de boot niet alleen mag laten}{Om je de boot anders kwijt kunt raken}{Omdat dat gezelliger is}

\question{5}{Waar moet je op letten als je een groot schip ziet verlijeren?}
\answerTextFour{Dat deze minder goed kan sturen}{Dat deze meer ruimte in beslag neemt}{Dat de boot plots kan opschuiven}{Dat er meer zuiging is}

\question{6}{Waar is de zuiging het ergst bij een groot schip?}
\answerTextFour{De voorkant}{De voor- en achterkant}{De zijkant}{De zij- en achterkant}

\question{7}{Wat gebeurt er wanneer de wind krimpt?}
\answerTextFour{Het gaat zachter waaien}{De wind draait tegen de richting van de klok in}{De wind draait met de richting van de klok mee}{De windvlagen worden zachter}

\question{8}{De dode hoek van een groot schip is}
\answerTextFour{Waar de schipper niets kan zien}{De hoek waar de meeste zuiging is}{De hoek die het schip met het water maakt}{De achterkant van het schip}

\chapter{Bruggen \& Sluizen}
\vspace{-120px}
\question{1}{Bij welke brug is doorvaart toegestaan en is tegenliggende vaart verboden?}


\begin{figure}[H]
	\centering
	\begin{minipage}[b]{0.23\textwidth}
		\includegraphics[width=\textwidth]{Hoofdstukken/Bruggen/pdf/brug_doorvaart_toegestaan.pdf}
		\centering
		A
	\end{minipage}
	\hfill
	\begin{minipage}[b]{0.23\textwidth}
		\includegraphics[width=\textwidth]{Hoofdstukken/Bruggen/pdf/brug_doorvaart_geen_tegenligger.pdf}
		\centering
		B
	\end{minipage}
	\hfill
	\begin{minipage}[b]{0.23\textwidth}
		\includegraphics[width=\textwidth]{Hoofdstukken/Bruggen/pdf/brug_doorvaart_verboden.pdf}
		\centering
		C
	\end{minipage}
	\hfill
	\begin{minipage}[b]{0.23\textwidth}
		\includegraphics[width=\textwidth]{Hoofdstukken/Bruggen/pdf/brug_aanstonds_toegestaan.pdf}
		\centering
		D
	\end{minipage}
\end{figure}


\question{2}{Welke brug gaat bijna open?}

\begin{figure}[H]
	\centering
	\begin{minipage}[b]{0.23\textwidth}
		\includegraphics[width=\textwidth]{Hoofdstukken/Bruggen/pdf/brug_aanstonds_toegestaan.pdf}
		\centering
		A
	\end{minipage}
	\hfill
	\begin{minipage}[b]{0.23\textwidth}
		\includegraphics[width=\textwidth]{Hoofdstukken/Bruggen/pdf/brug_sluitend.pdf}
		\centering
		B
	\end{minipage}
	\hfill
	\begin{minipage}[b]{0.23\textwidth}
		\includegraphics[width=\textwidth]{Hoofdstukken/Bruggen/pdf/brug_toegestaan.pdf}
		\centering
		C
	\end{minipage}
	\hfill
	\begin{minipage}[b]{0.23\textwidth}
		\includegraphics[width=\textwidth]{Hoofdstukken/Bruggen/pdf/brug_doorvaart_toegestaan.pdf}
		\centering
		D
	\end{minipage}
\end{figure}

\question{3}{Wat betekenen de groene borden op deze burg?}
\begin{figure}[H]	
	\vspace{-10px}
	\begin{minipage}[]{0.70\textwidth}
		\begin{enumerate}[topsep=0pt, label=\Alph*.]
			\item Brug buiten bediening
			\item Het gebied tussen de borden is het aangeraden vaargebied
			\item Doorvaart toegestaan
			\item Doorvaart toegestaan, tegenliggers mogelijk
		\end{enumerate}
	\end{minipage}
	\begin{minipage}[]{0.29\textwidth}
		\begin{figure}[H]
			\includegraphics[width=0.80\textwidth,right]{Hoofdstukken/Bruggen/pdf/brug_aanbevolen_gebied.pdf}
		\end{figure}
	\end{minipage}
	\vspace{-10px}
\end{figure}

\question{4}{Wat betekent een enkele gele ruit?}
\answerTextFour{Doorvaart verboden}{Doorvaart toegestaan, tegenliggers mogelijk}{Doorvaart toegestaan, tegenliggers niet mogelijk}{Brug buiten bediening}

\question{5}{De brug gaat al naar beneden. Mag je er nog onder door varen?}
\begin{figure}[H]	
	\vspace{-10px}
	\begin{minipage}[]{0.70\textwidth}
		\begin{enumerate}[topsep=0pt, label=\Alph*.]
			\item Ja, het groene licht brandt nog
			\item Ja, zolang het nog past
			\item Nee, \textit{tenzij} je niet meer kan stoppen
			\item Nee, je mag nooit onder een sluitende brug door varen
		\end{enumerate}
	\end{minipage}
	\begin{minipage}[]{0.29\textwidth}
		\begin{figure}[H]
			\includegraphics[width=0.80\textwidth,right]{Hoofdstukken/Bruggen/pdf/brug_sluitend.pdf}
		\end{figure}
	\end{minipage}
	\vspace{-10px}
\end{figure}

\chapter{Reglementen \& Voorrangsregels}
\vspace{-120px}
\section*{Wie heeft er voorrang? Denk ook goed na waarom en vul de letter in!}
\begin{table}[h!]
\centering
\begin{tabular}{l|l|l|l|l|l|l|l|l|l|l|l}
\textbf{1} & \textbf{2} & \textbf{3} & \textbf{4} & \textbf{5} & \textbf{6} & \textbf{7} & \textbf{8} & \textbf{9} & \textbf{10} & \textbf{11} & \textbf{12} \\ \hline
 \hspace{0.5 cm} & \hspace{0.5 cm}  & \hspace{0.5 cm} & \hspace{0.5 cm} & \hspace{0.5 cm} & \hspace{0.5 cm} & \hspace{0.5 cm} & \hspace{0.5 cm} & \hspace{0.5 cm} & \hspace{0.5 cm} & \hspace{0.5 cm} & \hspace{0.5 cm}
\end{tabular}
\end{table}
\begin{figure}[h!]
    \centering
    \includegraphics[width=\textwidth]{Hoofdstukken/Oefenvragen/pdf/regelementen_1.pdf}
\end{figure}

\newpage
\begin{table}[h!]
	\centering
	\begin{tabular}{l|l|l|l|l|l|l|l|l|l|l}
		\textbf{13} & \textbf{14} & \textbf{15} & \textbf{16} & \textbf{17} & \textbf{18} & \textbf{19} & \textbf{20} & \textbf{21} & \textbf{22} & \textbf{23} \\ \hline
		\hspace{0.5 cm} & \hspace{0.5 cm}  & \hspace{0.5 cm} & \hspace{0.5 cm} & \hspace{0.5 cm} & \hspace{0.5 cm} & \hspace{0.5 cm} & \hspace{0.5 cm} & \hspace{0.5 cm} & \hspace{0.5 cm} & \hspace{0.5 cm}
	\end{tabular}
\end{table}
\begin{figure}[h!]
    \centering
    \includegraphics[width=\textwidth]{Hoofdstukken/Oefenvragen/pdf/regelementen_2.pdf}
\end{figure}
\newpage

\question{24}{Wat is de betekenis van het bord in het figuur rechts?}
\vspace*{-0.5cm}
\answerTextPicture{In-, uit-, of doorvaren verboden}{Einde van een verbod of gebod}{Verboden geluidsseinen te maken}{Verplichting voor het bord stil te houden}{Hoofdstukken/Reglementen/pdf/B5.pdf}

\question{25}{Wat betekent het volgende geluidssein: \slong \sspace  \sshort \sspace  \slong }
\answerTextFour{Attentie}{Blijf weg sein}{Ik ga bakboord uit}{Verzoek tot bediening van brug of sluis}

\question{26}{Wat is de minimale leeftijd voor het varen op een groot schip?}
\answerTextFour{Geen minimale leeftijd}{12 jaar}{16 jaar}{18 jaar}

\question{27}{Geldt het BPR op alle Nederlandse binnenwateren?}
\answerTextFour{Ja}{Ja, met uitzondering van het IJsselmeer}{Ja, met uitzondering van de Rijn }{Nee}

\question{28}{Welke schepen zijn toegestaan volgens dit bord?}
\vspace*{-0.5cm}
\answerTextPicture{Snelle motorschepen}{Kleine schepen}{Recreatievaart}{Roei- en zeilschepen}{Hoofdstukken/Reglementen/pdf/E16.pdf}

\question{29}{Welke snelheid mag hier maximaal gevaren worden?}
\vspace*{-0.5cm}
\answerTextPicture{6 kilometer per uur}{6 mijl per uur}{6 zeemijl per uur}{6 knopen}{Hoofdstukken/Reglementen/pdf/B6.pdf}

\question{30}{Mag iedereen in een snelle motorboot varen (>20km/h)?}
\answerTextFour{Ja, hier zijn geen vereisten voor}{Nee, je moet minimaal 16 jaar zijn}{Nee, je moet minimaal 18 jaar zijn}{Nee, je moet minimaal 18 jaar zijn en je KVB I of II hebben}

\chapter{Reglementen \& Voorrangsregels}
\vspace{-120px}
\section*{Wie heeft er voorrang? Denk ook goed na waarom en vul de letters in!}
\begin{table}[h!]
\centering
\begin{tabular}{l|l|l|l|l|l}
\textbf{1} & \textbf{2} & \textbf{3} & \textbf{4} & \textbf{5} & \textbf{6} \\\hline
 \hspace{0.5 cm} & \hspace{0.5 cm}  & \hspace{0.5 cm} & \hspace{0.5 cm} & \hspace{0.5 cm} & \hspace{0.5 cm} 
\end{tabular}
\end{table}
\begin{figure}[h!]
    \centering
    \includegraphics[width=\textwidth]{Hoofdstukken/Oefenvragen/pdf/regelementen.pdf}
\end{figure}

\chapter{Schiemannen}
\vspace{-120px}
\question{1}{Met welke knoop maak je twee touwen van gelijke dikte aan elkaar vast?}
\answerTextFour{Schootsteek}{Paalsteek}{Constrictor}{Platte knoop}

\question{2}{Welke knopen zie je in de afbeelding hiernaast?}
	\vspace{-20px}
\begin{figure}[H]	

	\begin{minipage}[]{0.59\textwidth}
		\begin{enumerate}[topsep=0pt, label=\Alph*.]
			\item Links: platte knoop, rechts: een halve steek
			\item Links: achtknoop, rechts: schootsteek
			\item Links: reefsteek, rechts: dubbele halve steek
			\item Links: platte knoop, rechts: dubbele schootsteek
		\end{enumerate}
	\end{minipage}
	\begin{minipage}[]{0.20\textwidth}
	\begin{figure}[H]
		\includegraphics[width=\textwidth,right]{Hoofdstukken/Schiemannen/pdf/platteknoop.pdf}
	\end{figure}
\end{minipage}
	\begin{minipage}[]{0.20\textwidth}
		\begin{figure}[H]
			\includegraphics[width=\textwidth,right]{Hoofdstukken/Schiemannen/pdf/dubbele_schootsteek.pdf}
		\end{figure}
	\end{minipage}
	\vspace{-20px}
\end{figure}

\question{3}{Welke knoop laat een dubbele halve steek zien?}

\answerPicture{Hoofdstukken/Schiemannen/pdf/dubble_halve_steek_slippend.pdf}{Hoofdstukken/Schiemannen/pdf/paalsteek.pdf}{Hoofdstukken/Schiemannen/pdf/dubble_halve_steek.pdf}{Hoofdstukken/Schiemannen/pdf/halve_steek.pdf}



\question{4}{Met welke knoop leg je een niet-slippende lus in een lijn?}
\answerTextFour{Schootsteek}{Paalsteek}{Dubbele slipsteek}{Dubbele halve steek}

\question{5}{Welke stelling is waar?}
\answerTextFour{Natuurvezel kan beter tegen UV en heeft een hogere breeksterkte dan kunstvezel}{Kunstvezel heeft minder rek en is gevoeliger voor schavielen dan natuurvezel}{Natuurvezel is slijtvaster en heeft minder rek dan kunstvezel}{Kunstvezel kan beter tegen UV en is slijtvaster dan natuurvezel}

\question{6}{Bij welke lijn is enige mate van rek gunstig?}
\answerTextFour{Landvasten en ankerlijn}{Vallen}{Schoten}{Rijglijn}


\chapter{Krachten op het schip}
\vspace{-120px}
\question{1}{Met welk zeil loef je op?}
\begin{enumerate}[topsep=0pt, label=\Alph*.]
	\item De fok
	\item Het grootzeil
\end{enumerate}


\question{2}{Wat is er fout aan het zeil in het figuur hiernaast?}
	\vspace{-20px}
\begin{figure}[H]		
	\begin{minipage}[]{0.70\textwidth}
		\begin{enumerate}[topsep=0pt, label=\Alph*.]
			\item Het zeil staat te strak
			\item Het zeil staat te los
			\item Het zeil is slecht gehesen
			\item Het zeil is kapot
		\end{enumerate}
	\end{minipage}
	\begin{minipage}[]{0.29\textwidth}
		\begin{figure}[H]
			\includegraphics[width=0.80\textwidth,right]{Hoofdstukken/Krachten/pdf/zeil_los.pdf}
		\end{figure}
	\end{minipage}
	\vspace{-20px}
\end{figure}



\question{3}{Je vaart halve wind en wil snel oploeven, wat doe je?}
\answerTextFour{Beide zeilen vieren}{Je fok vieren en grootzeil aantrekken}{Je grootzeil vieren en fok aantrekken }{Beide zeilen aantrekken}

\question{4}{Hoe controleer je of je grootzeil goed staat?}
\answerTextFour{Door je zeil te laten vieren}{Door deze te laten vieren tot hij tegenbolt en daarna een klein beetje aan te trekken}{Door een beetje af te vallen}{Door naar je zeillatjes te kijken}

\question{5}{Hoe kan je zien dat je zeil te los staat?}
\answerTextFour{Je gaat sneller varen}{Het voorlijk van je zeil gaat tegenbollen}{Je fok bolt tegen}{Het achterlijk van je zeil gaat klapperen}
\chapter{Antwoorden}
\vspace{-120px}
%%%% Hoofdstuk 1 %%%%%
\begin{table}[h]
	\centering
	\begin{tabular}{c|c|c|m{9.5cm}}
	\textbf{Hfd.}       & \textbf{Vraag} & \textbf{Antwoord} & \textbf{Toelichting}                                             \\ \hline
	\multirow{11}{*}{\sffamily\bfseries{\textcolor{ocre}{\LARGE1}} } & 1  & B & \\ \cline{2-4}          
	& 2 & B &  \\ \cline{2-4} 
	& 3 & A &  \\ \cline{2-4} 
	& 4 & D &  \\ \cline{2-4} 
	& 5 & A &  I: Hogerwal, II: Hoge kant, III: lage kant en IV: lagerwal\\ \cline{2-4} 
	& 6 & A &  \\ \cline{2-4} 
	& 7 & C &  A Aan de wind, B voor de wind, C ruime wind,   D halve wind\\ \cline{2-4} 
	& 8 & B &  A Ruime wind,  B aan de wind,  C voor de wind, D halve wind\\ \cline{2-4} 
	& 9 & D &  A Aan de wind, B ruime wind,   C halve wind,   D voor de wind\\ \cline{2-4} 
	& 10 & A &  \\ \cline{2-4} 
	& 11 & C & \\ \hline
	\multirow{4}{*}{\sffamily\bfseries{\textcolor{ocre}{\LARGE2}} } & 1   & B         &  \\ \cline{2-4} 
	& 2 & B &  \\ \cline{2-4} 
	& 3 & D &  \\ \cline{2-4} 
	& 4 & A & \\ \hline
	\multirow{6}{*}{\sffamily\bfseries{\textcolor{ocre}{\LARGE3}} } & 1   & A         & Zeilboot gaat voor spierkracht gaat voor motorboot \\ \cline{2-4} 
	& 2 & B &  Stuurboordswal gaat voor\\ \cline{2-4}  
	& 3 & A &  Zeilboten onderling: Zeilen over bakboord gaan voor\\ \cline{2-4} 
	& 4 & A &  Zeilboten onderling: Loef wijkt voor lij \\ \cline{2-4} 
	& 5 & C &  Zeilboot gaat voor spierkracht gaat voor motorboot \\ \cline{2-4} 
	& 6 & A &  Grote schepen gaan voor op kleine schepen\\ \hline 
	\multirow{6}{*}{\sffamily\bfseries{\textcolor{ocre}{\LARGE4}} } & 1 & A & Platteknoop voor gelijke dikte, schootsteek voor ongelijke dikte\\ \cline{2-4} 
	& 2 & B &  \\ \cline{2-4} 
	& 3 & D & Met een slipsteek kun je een mastworp `borgen' \\ \cline{2-4} 
	& 4 & D & A Mastworp, B Dubbele halve steek, C Slipsteek, D Halve steek\\ \cline{2-4} 
	& 5 & D &  \\ \cline{2-4} 
	& 6 & B &  \\ \hline 
	\multirow{5}{*}{\sffamily\bfseries{\textcolor{ocre}{\LARGE5}} } & 1 & B &  \\ \cline{2-4} 
	& 2 & B & Het zeil bolt tegen, dus het staat te los \\ \cline{2-4} 
	& 3 & B & Het fok valt af en het zeil loeft op \\ \cline{2-4} 
	& 4 & B & \\ \cline{2-4} 
	& 5 & D & 
	\end{tabular}
\end{table}


\part{Zeil manoeuvres}
\label{part:manoeuvre}
\usechapterimagefalse
\chapter{Oploeven}
\vspace{-120px}

\begin{tikzpicture}[remember picture,overlay]
\node[inner sep=0pt] (background) at (current page.center) {\includegraphics[width=\paperwidth, page = 1]{Hoofdstukken/Manoeuvres/manoeuvres.pdf}};
\end{tikzpicture}


\chapter{Afvallen}
\vspace{-120px}

\begin{tikzpicture}[remember picture,overlay]
\node[inner sep=0pt] (background) at (current page.center) {\includegraphics[width=\paperwidth, page = 2]{Hoofdstukken/Manoeuvres/manoeuvres.pdf}};
\end{tikzpicture}
\usechapterimagefalse
\chapter{Overstag}
\vspace{-120px}
\begin{tikzpicture}[remember picture,overlay]
\node[inner sep=0pt] (background) at (current page.center) {\includegraphics[width=0.95\paperwidth, page = 3]{../Hoofdstukken/Manoeuvres/Manoeuvres.pdf}};
\end{tikzpicture}

\usechapterimagefalse
\chapter{Gijp}
\vspace{-120px}
\begin{tikzpicture}[remember picture,overlay]
\node[inner sep=0pt] (background) at (current page.center) {\includegraphics[width=0.95\paperwidth, page = 4]{../Hoofdstukken/Manoeuvres/Manoeuvres.pdf}};
\end{tikzpicture}
\usechapterimagefalse
\chapter{Hogerwal}
\vspace{-120px}

\begin{tikzpicture}[remember picture,overlay]
\node[inner sep=0pt] (background) at (current page.center) {\includegraphics[width=\paperwidth, page = 5]{../Hoofdstukken/Manoeuvres/Manoeuvres.pdf}};
\end{tikzpicture}
\newpage
\thispagestyle{plain}
\begin{tikzpicture}[remember picture,overlay]
\node[inner sep=0pt] (background) at (current page.center) {\includegraphics[width=\paperwidth, page = 6]{../Hoofdstukken/Manoeuvres/Manoeuvres.pdf}};
\end{tikzpicture}
\usechapterimagefalse
\chapter{Man over boord}
\vspace{-120px}

\begin{tikzpicture}[remember picture,overlay]
\node[inner sep=0pt] (background) at (current page.center) {\includegraphics[width=0.95\paperwidth, page = 7]{Hoofdstukken/Manoeuvres/manoeuvres.pdf}};
\end{tikzpicture}
\usechapterimagefalse
\chapter{Stormrondje}
\vspace{-120px}

\begin{tikzpicture}[remember picture,overlay]
\node[inner sep=0pt] (background) at (current page.center) {\includegraphics[width=0.95\paperwidth, page = 8]{../Hoofdstukken/Manoeuvres/Manoeuvres.pdf}};
\end{tikzpicture}
\usechapterimagefalse
\chapter{Lagerwal}
\vspace{-120px}

\begin{tikzpicture}[remember picture,overlay]
\node[inner sep=0pt] (background) at (current page.center) {\includegraphics[width=\paperwidth, page = 9]{Hoofdstukken/Manoeuvres/Manoeuvres.pdf}};
\end{tikzpicture}
\newpage


\newpage\null\thispagestyle{empty}\newpage
\newpage\null\thispagestyle{empty}\newpage
\newpage\null\thispagestyle{empty}\newpage

\newpage
\begingroup
\thispagestyle{empty}
\begin{tikzpicture}[remember picture,overlay]
\node[inner sep=0pt] (last_page) at (current page.center) {\includegraphics[width=\paperwidth, page = 2]{\omslag}};
\end{tikzpicture}
\vfill
\endgroup

%----------------------------------------------------------------------------------------

\end{document}
